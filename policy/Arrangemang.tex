\section{Arrangemangspolicy}
Denna policy gäller alla arrangemang i Datateknologsektionens regi, samt övriga arrangemang i de lokaler som disponeras av Datateknologsektionen.

\subsection{Omfattning}
Alla arrangemang skall vara inkluderande för alla medlemmar på sektionen med ett fåtal undantag, som till exempel: phaddersittningar, mottagningsarrangemang och kommittetradtionella arrangemang. Arrangemang med en specifik målgrupp ska ha styrelsens godkännande.

\subsection{Hållbarhet}
Alla arrangemang ska sikta mot att minimera sin miljö- och klimatpåverkan.

Några exempel på sätt att minimera sin miljöpåverkan kan vara att:
\begin{itemize}
    \item I största utsträckning använda återanvändbara tallrikar, bestick och glas.
    \item Källsortera rester från arrangemanget.
    \item I största mån köpa svenska och ekologiska råvaror.
\end{itemize}

\subsection{Allergier}
Om ett evenemang har föranmälan ska alla med anmälda allergier få likvärdig mat som övriga i den utsträckning det går. Om ingen föranmälan sker, t.ex. afterschool och pubar, ska veganskt och glutenfritt alternativ kunna erbjudas. Undantag kan ges vid t.ex. provningar.

\subsection{Alkohol och droger}
Arrangörer ska arrangera i enlighet med Chalmers Studentkårs policy \textit{Fest- och alkoholpo-
licy vid Chalmers Studentkår.}