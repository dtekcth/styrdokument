\documentclass{dtek}
\usepackage{csquotes}
\usepackage{titlesec}
\setcounter{tocdepth}{1}

\title{Policys för Datateknologsektionen}

\makeheadfoot

\begin{document}
\maketitle
Dessa policys är satta av Datateknologsektionens styrelse för att komplementera sektionens stadgar och reglemente, samt kårens centrala stadgar, reglemente och policys.

Frågor gällande policys hänvisas till styret@dtek.se eller via post.

\tableofcontents

\newpage
\section{Arrangemangspolicy}
Denna policy gäller alla arrangemang i Datateknologsektionens regi, samt övriga arrangemang i de lokaler som disponeras av Datateknologsektionen.

\subsection{Omfattning}
Alla arrangemang skall vara inkluderande för alla medlemmar på sektionen med ett fåtal undantag, som till exempel: phaddersittningar, mottagningsarrangemang och kommittetradtionella arrangemang. Arrangemang med en specifik målgrupp ska ha styrelsens godkännande.

\subsection{Hållbarhet}
Alla arrangemang ska sikta mot att minimera sin miljö- och klimatpåverkan.

Några exempel på sätt att minimera sin miljöpåverkan kan vara att:
\begin{itemize}
    \item I största utsträckning använda återanvändbara tallrikar, bestick och glas.
    \item Källsortera rester från arrangemanget.
    \item I största mån köpa svenska och ekologiska råvaror.
\end{itemize}

\subsection{Allergier}
Om ett evenemang har föranmälan ska alla med anmälda allergier få likvärdig mat som övriga i den utsträckning det går. Om ingen föranmälan sker, t.ex. afterschool och pubar, ska veganskt och glutenfritt alternativ kunna erbjudas. Undantag kan ges vid t.ex. provningar.

\subsection{Alkohol och droger}
Arrangörer ska arrangera i enlighet med Chalmers Studentkårs policy \textit{Fest- och alkoholpo-
licy vid Chalmers Studentkår.}
\newpage
\section{Basenpolicy}
Alla medlemmar i Datateknologsektionen har fri tillgång till sektionslokalen Basen. Därför har en policy för användande av Basen tagits fram.

\subsection{Nyttjande av lokalen}
Vid nyttjande av lokalen råder skötsel-, ordning- och städbestämmelser.

\subsubsection{Skyldigheter}
Medlem skall
\begin{itemize}
    \item följa sektionens och kårens styrdokument, samt svensk lag.
    \item Städa efter sig själv.
    \item ej störa andra.
    \item ej avlägsna objekt från Basen utan styrelsens godkännande.
    \item ej förvara stora och klumpiga saker i Basen eller dess närhet.
    \item medverka i Nollstäd.
    \item ansvara för sina gäster.
\end{itemize}

\subsubsection{Undantag}
\begin{itemize}
    \item Vid arrangemang begränsas ytan för gemene teknolog till köket. Vid vissa arrangemang är även vistelse i köket förbjudet, exempelvis under serveringstillstånd.
    \item Under längre skoluppehåll är Basen avstängd, då resurser inte finns för att städa den.
    \item Delar av, eller hela Basen kan efter styrelsebeslut stängas ner av andra anledningar.
\end{itemize}

\subsection{Fest i lokalen}
Alla fester i Basen skall följa nedanstående regler. Det räknas som fest när fyra eller fler dricker alkohol samtidigt i Basen. Ansvariga har rätt till att avvisa folk från lokalen som inte följer dessa regler. Datateknologer har dock alltid rätt att få tillgång till Basens kök.
\subsubsection{Regler för fest i Basen}
\begin{itemize}
    \item En ansvarig ska finnas på plats, som alltid hålla sig vid sina sinnens fulla bruk och ha uppsikt över hela lokalen och dess omgivning.
    \item Man får ej bruka alkohol i Basen mellan 03:00 och 17:00 på helgfria vardagar.
    \item Det är bara sektionsmedlemmar som kan vara ansvariga.
    \item Köket skall vara fritt från fest.
    \item Brandgränsen skall alltid följas(70 personer i \textbf{hela} lokalen).
    \item Efter 22:00 ska fönsterna vara stängda om nästkommande dag är en vardag.
\end{itemize}

\subsubsection{Paxa basen för arr}
För att utföra ett planerat arr i Basen så ska man skicka in en in paxning via Basenpax formuläret. Beslut om paxningar tas upp nästkommande styrelsemöte.
Efter godkännade från styrelsen så ska arrangemanget aktivitetanmälas på studentportalen. Där ska även den ansvariga skrivas upp.

\subsubsection{Spontan fest}
Om du öppnar upp för fest i basen ska detta anmälas direkt vilket man gör via fest@dtek.se.
I mailet anger man vem som står som ansvarig. Om ni känner er osäkra på om en anmälan
krävs, gör det hellre en gång för mycket än för lite då Styrelsen och DRust vill vilka som
vistats i Basen i fall något händer. 
När festen är klar behöver man städa Basen enligt arrstädslappen.

\subsubsection{Spontanöppning av flipperrummet}
Flipperrummet kan spontanöppnas för fest om Basen inte är bokad och inte redan är öppnad.
Detta görs genom att maila vem som är ansvarig till fest@dtek.se, på samma sätt som när man öppnar hela Basen.
Efteråt ska man städa flipper enligt flipperstädlappen.

\subsubsection{Överlåta festen}
Om du öppnar basen är du första ansvarig att se till att det blir städat. Om du överlåter
festen till någon annan måste personen som tar över ansvaret skicka in ett nytt mail till
fest@dtek.se där den anger att den tar över.

\subsubsection{Städ}
Om man har arrangerat i Basen så ska det städas enligt de instruktioner som finns på
arrstäd-lapparna hängandes i Basens städskrubb. Ifall att någon annan varit i köket och
har lagat mat eller arrangerat skall detta anges, annars är det ert ansvar att det är städat.
Ett mail ska skickas till fest@dtek.se där man säger till att Basen stängs ner, samt bifogar
bilder från Puben, Köket, Hallen och Flipperrummet. Dessa är till för att skydda dig som
ansvarig från skuld. Städningen ska vara avklarade innan 08:00 följande morgon. Om det
inte är städat innan dess eller om DRust alternativt Styret anser att du inte har städat
eller fyllt i hela listan kommer städningen inte att godkännas och kan leda till påföljder.

\subsection{Åtgärder}
Om några regler bryts kommer åtgärder att vidtas. Du som ansvarig kommer att bli med-
delad att ärendet kommer lyftas på ett styrelsemöte. Styrelsen har rätt att bestämma vilka
åtgärder som kommer tas.

\newpage
\section{Aspningspolicy}
Alla kommittéer på sektionen har rätt till att hålla i en aspning. Tanken med aspningen är att visa asparna vad kommitténs arbete går ut på och ge dem en chans att testa på arbetet. Fokus ska ligga på verksamheten. För att se till att alla kommittéer har samma förutsättningar och att alla aspar behandlas på ett värdigt sätt har denna aspningspolicy tagits fram. 

\subsection{Innan Aspningen}
\begin{itemize}
    \item Eftersom det är väldigt många kommittéer som arrangerar aspning samtidigt är det viktigt att sätta sitt aspschema så tidigt som möjligt. Det kommer vara omöjligt att helt undvika dubbelbokningar men se till att undvika det i största mån. 
    \item Innan aspningen börjar är det kul med aspaffischer, dessa kan sättas upp i Basen men styrelsen har rätt att plocka ner aspaffischer som anses vara opassande. Riktlinjer för aspaffischer är
    \begin{itemize}
        \item Alkohol får inte vara fokus av affischen.
        \item Affischer får inte innehålla något form av kränkande material. 
    \end{itemize}
\end{itemize}
\subsubsection{Aspningsplan}
Innan aspningsperioden böjar ska kommittéer skriva ner en aspningsplan som innehåller arr som ska hållas, en beskrivning av dem och en ungefärlig kostnadsberäkning och sedan skicka detta till styrelsen. 


\subsection{Under Aspningen}
\subsubsection{Förhållningsregler}
\begin{itemize}
    \item Det är inte tillåtet att lyfta sin egen kommitté på bekostnad av en annan kommitté. 
    \item Det är inte tillåtet att avråda aspar från att aspa specifika kommittéer. 
    \item Aspningstillfällen skall marknadsföras i enlighet med sektionens kommunikationspolicy, och alla skall uppmuntras lika mycket till att söka eller aspa.
    \item Ingen alkoholhets får förekomma. 
    \item Om alkohol erbjuds vid ett aspningsarrangemang så skall även motsvarande alkoholfria alternativ erbjudas.
    \item Arrangörer har rätt att avvisa aspar som missköter sig från pågående asparrangemang, men endast styrelsen har rätt att stänga av aspar från aspningen i helhet.
    \item Person som arrangerar aspning eller medverkar på aspning av annan anledning än att aspa får ej missbruka sin maktposition på något sätt.
    \item Person som arrangerar aspning ska avstå från att inleda intim relation med asp.
    \item Aspar får inte favoriseras eller särbehandlas.
    \item Aspplagg ska vara roliga att ha på sig och inte vara förnedrande för den som bär det. 
\end{itemize}
\subsubsection{Arrangemang}
\begin{itemize}
    \item Vid varje arrangemang ska det finnas minst två stycken nyktra arrangerande som har ansvar. 
    \item Platsbegränsade arrangemang bör undvikas då alla ska ha möjlighet att delta under aspningen.
    \item Asptillfällen ska vara så billiga som möjligt att delta på, men det är okej för aspar att betala en mindre summa för mat om det skulle behövas. 
    \item Det ska meddelas i förväg ifall mat serveras eller ej under ett asptillfälle. 
    \item Asptillfällen ska ej ha fokus på alkohol, utan ska fokusera på kommitténs verksamhet.
\end{itemize}


\subsection{Efter Aspningen}
Efter aspningen är slut ska alla aspar fortsatt behandlas lika och ingen favorisering får förekomma även efter nominering från valberedningen. 
\subsubsection{Valberedning}
\begin{itemize}
    \item Varje kommitté har rätt att be om valberedning, valberedningens sammansättning beskrivs i reglementet. 
    \item Firande av/festande med nominerad grupp i egenskap av nominering får ej förekomma. 
    \item Det ska vara tydligt för de icke-nominerade att de fortfarande har möjlighet att söka kommittén på sektionsmötet. 
\end{itemize}


\newpage


\section{Valberednings- och Invalspolicy}

Denna policy beskriver hur valberedningens process ska genomföras samt hur inval till sektionens kommittéer ska ske. Alla medlemmar har rätt att söka alla förtroendeposter inom sektionen.

\subsection{Valberedning} Alla kommittéer har rätt att begära valberedning från styrelsen inför inval av nya kommittemedlemmar. Valberedningens giltighet definieras i sektionens reglemente. Valberedningen har tystnadsplikt gällande diskussioner och detaljer om de sökande. Endast en sammanfattande bedömning av de nominerade ska presenteras vid invalsmötet.

\subsection{Personinval} Inför inval till en kommitté ska intresserade medlemmar anmäla sitt intresse till sektionsstyrelsen senast 10 dagar före det sektionsmöte där invalet ska ske. Detta gäller även fyllnadsval. Anmälan är dock inte bindande.

En lista med alla de sökande som är intresserade av en förtroendepost ska publiceras i sektionens lokaler senast 10 dagar innan det aktuella mötet.

En sektionsmedlem har fortfarande rätt att söka en post även om denne inte anmält sig i förväg. Vid invalsmötet ska sektionsstyrelsen dock tydligt klargöra vilka av de sökande som har anmält sig inom den angivna tidsfristen.


\newpage
\section{Representationspolicy}
Den här policyn berör alla medlemmar i Datateknologsektionen, speciellt medlemmar som
bär kläder som förknippas med teknologsektionen, exempelvis kommitté-overaller. Policyn
syfte är att se till att sektionens medlemmar agerar föredömligt och värnar om sektionens
varumärke och varandra. Denna policy bör vara trivial för alla medlemmar men i fallet att
den inte är det så finns allt nerskrivet här.

\subsection{Värdigt och föredömligt beteende}
Varje medlem i sektionen förväntas agera och bete sig enligt sektionens och Chalmers
studentkårs värdegrunder och på ett sätt som inte försämrar datateknologsektionens ryk-
te. Ingen medlem får utföra någon form av diskriminering, hets, eller tvång. Medlemmar
förväntas visa respekt för varandra, andra sektioner, högskolor och övriga medmänniskor.

\subsection{Sektionsfärg}

Datateknologsektionens officiella färg är orange. Det finns ingen officiell färgkod.

För webdesign rekommenderas färgkoden \#F97316 (Tailwind Orange 500). För tryck hos tryckeri och beställning av märken rekommenderas det att välja en färg från företagets egna lista av färger. 
 
För märken används ibland en neonorange färg, observera att vissa färgkoder kan upplevas bruna på fysiska märken. För representationskläder är alla varianter av orange välkomna. 

\subsection{Sektionslogotyp}
Den officiella Sektionslogotypen återfinnes via länken \url{https://styrdokument.dtek.se/loggor}.
\newpage
\section{Kommunikationspolicy}
Arrangemang riktade mot samtliga sektionsmedlemmar samt uttalanden från sektionen och dess kommitter skall omfattas av policyns bestämmelser. Policyns syfte är att försäkra gemene datateknolog tillgång till sektionsviktig information.

\subsection{Annonsering}
Annonsering, eller i folkmun PR, skall följa annonseringskrav för att skapa kontinuitet
och en igenkänningsförsäkran för datateknologer. Annonseringskraven bidrar även till en
rättvis behandling av samtliga datateknologer. Ytterligare gör detta att sektionens olika
kommittéer och dess verksamheter blir mer synliga för gemene teknolog.

\subsubsection{Allmänna råd för annonsering}
Annonsering behöver inte nödvändigtvis göras enbart då arrangemangets alla detaljer är
fastställda. Ett allmänt råd är att annonsera datum och annan information som fastställts
trots eventuell avsaknad av information.

\subsubsection{Medium för annonsering}
Annonsering av arrangemang skall tillgängliggöras via minst två av följande:
\begin{itemize}
    \item Sektionens nyhetsbrev.
    \item Anslagstavlan i Basen.
    \item Evenemang på Facebook med publicering i sektionens officiella Facebook-grupp.
\end{itemize}
Utöver ovannämnda medium uppmuntras arrangörer, i största möjliga utsträckning, att
tillgängliggöra informationen på ytterligare medium som till exempel data-kalendern, kom-
mittesidor och eller andra sociala medier.

\subsubsection{Innehållskrav för annonsering}
All nödvändig information om arrangemanget ska finnas tillgänglig på engelska. Annonser
bör, i största möjliga utsträckning, informera om ifall svenskakunskap är rekommenderat
för deltagare, exempelvis på sittningar ”toastade” på svenska.

Annonser bör innehålla följande information om:
\begin{itemize}
    \item Datum, tid och plats för arrangemanget.
    \item Priser, ifall tillämpningsbart, för att delta vid arrangemanget.
\end{itemize}
I sådana fall där innehållskravet ej kan uppnås som t.ex. om pris eller lokal ej är fastställd
bör det istället framgå hur man förväntas bli informerad.

\subsubsection{Övriga krav för annonsering}
Den som omfattas av policyn bör annonsera arrangemanget i god tid, i regel bör man
annonsera arrangemanget minst sju dagar i förväg. 14 dagar är att anse som mycket god
tid. Detta för att försäkra att informationen hinner nå så många som möjligt. Det kan vara
anmärkningsvärt att sektionsmedlemmar spenderar olika mycket tid på sociala medier och
har olika långt till sektionens lokaler.

\subsection{Kungörelser eller andra underrättelser}
För meddelanden som är av en sådan vikt att deras allmänna spridning på sektionen anses
önskvärd bör kravet för att informationen nås ut till sektionens medlemmar anses vara
högre. Sektionsmöteskallelser omfattas av denna bestämmelse.

Meddelanden som omfattas av denna punkt bör publiceras på alla av följande:
\begin{itemize}
    \item Sektionens nyhetsbrev.
    \item Anslagstavlan i Basen.
    \item Evenemang på Facebook med publicering i sektionens officiella Facebook-grupp.
\end{itemize}
Datum för sektionsmöten bör, i de fall det är tillämplingsbart, tillgängliggöras direkt, även
om den formella kallelsen ej är tillgänglig vid tillfället.

\subsection{Styrelsens kommunikation}
Då styrelsen önskar nå ut till specifika datateknolog(er) så bör e-post användas om ingen
annan känd och effektivare metod existerar. Vid de tillfällen då individens e-post inte är
känd så kommer högskolans tillhandahållna student e-post att användas.

\subsection{Kommunikationskanaler}
Här behandlas sektionen flertal officiella kommunikationskanaler. Ej listade här är kom-
mitteernas egna kommunikationskanaler.

Sektionens officiella kommunikationskanaler:
\begin{itemize}
    \item Sektionens Facebook-grupp,\\
Datateknologsektionen Chalmers,\\
www.fb.com/groups/cth.dtek
    \item Sektionens Nyhetsbrev på e-post,\\
Veckobrevet,\\
www.dtek.se/newsletter
    \item Anslagstavla (övre korridor),\\
Sektionens Anslagstavla,\\
I Basen-korridoren, tavlan höger om Undercentral-dörren
    
    \item Anslagstavla (nedre korridor),\\
Sektionens Anslagstavla,\\
I Basen-korridoren, tavlan efter korridorstrapporna
    \item Sektionens Twitter-sida,\\
Datateknologsektionen Chalmers,\\
www.twitter.com/datachalmers
    \item Sektionens Facebook-sida,\\
Datateknologsektionen,\\
www.facebook.com/dtekchalmers

\end{itemize}

\newpage
\section{Personuppgiftspolicy för uppdragstagare}
Använd den här policyn som en guide för att hjälpa dig i ditt GDPR-arbete. Om du har
några frågor eller funderingar, kontakta Styrelsen: Styrelsen@dtek.se.
Den här policyn skapades 2018 och uppdaterades 2020 som ett led i arbetet med att göra
sektionen, dess kommittéer och intresseföreningar kompatibla med GDPR och ska om allt
fungerar som det är tänkt uppdateras löpande.

\subsection{Om GDPR och ideellt engagemang}
Det är viktigt att det är lätt och roligt att engagera sig ideellt på datateknologsektionen,
även när vi behöver hålla oss till gällande lagar och regler. Därför försöker vi hålla admi-
nistrationen kring GDPR till ett minimum. Följande behöver dock utföras för att vi ska
följa lagen:

\subsection{Vid inval}
Den nyinvalde ska så snart som möjligt skriva på ett kontrakt för att bli personuppgifts-
behandlare, vilket innebär att den får samla in och hantera personuppgifter för Datatek-
nologsektionen.

\subsection{Vid inaktivitet}
Om en person slutar vara aktiv i en kommitté eller intresseförening förlorar den åtkomst
till kommitténs personuppgifter tills den blir aktiv igen, detta för att minimera antalet
personer med tillgång till personuppgifter.

\subsection{Tredje parts åtkomst}
När personuppgifter delas med parter som ej är personuppgiftsbehandlare på Datatekno-
logsektionen är de att anse som tredje part. Det förekommer att personuppgifter delas med
tredje parter i olika delar av sektionens verksamhet.

\subsubsection{Huvudregel}
En tredje part kan få åtkomst till personuppgifter om det föreligger särskilda skäl, förutsatt
att detta godkänns av personuppgiftsansvarige i Styrelsen. Vid sådana omständigheter ska
personen i fråga även underteckna ovan nämnda kontrakt.

%5.4.1 borde hrefas
\subsubsection{Undantag}
Undantag från 5.4.1 är då personuppgiftsbehandlare nyttjar tredje parts tjänster på inter-
net. Då skall personuppgiftsbehandlaren försäkra att tjänsten har en personuppgiftspolicy som är enligt med GDPR och ej delar sektionens personuppgifter med annan tredje part
som ej uppfyller huvud- eller undantagsregeln.

Vid samarrangemang bör en personuppgiftsansvarig utses i den sektionskommittée som ar-
rangerar med en tredje part. Dess uppgift är att säkerställa att den arrangerade kommittén
hanterar personuppgifter som sektionen anser som lämpligt för att undantaget skall kunna
tillämpas.

\subsection{Dokumentation}
Styrelsen ska upprätthålla en lista över alla personer med åtkomst till personuppgifter,
samt i vilka sammanhang de kommer åt dessa personuppgifter. Exempel på sammanhang
är \enquote{Phixare i D6}.

\subsection{Behandling av personuppgifter}
GDPR är hårdare kring lagring av personuppgifter än den tidigare personuppgiftslagen
(PUL). Nedan följer praktiska riktlinjer för hur personuppgifter ska hanteras med nya
lagstiftningen.

\subsubsection{Grundprinciper}
GDPRs grundprinciper är att man som personuppgiftsansvarig:
\begin{itemize}
    \item Måste ha stöd i dataskyddsförordningen för att få behandla personuppgifter.
    \item Bara får samla in personuppgifter för specifika, särskilt angivna och berättigade ändamål.
    \item Inte ska behandla fler personuppgifter än vad som behövs för ändamålen.
    \item Ska se till att personuppgifterna är riktiga.
    \item Ska radera personuppgifterna när de inte längre behövs.
    \item Ska skydda personuppgifterna, till exempel så att inte. obehöriga får tillgång till dem och så att de inte förloras eller förstörs.
    \item Ska kunna visa att och hur ni lever upp till dataskyddsförordningen.
\end{itemize}

Exempel på personuppgifter:
\begin{itemize}
    \item Ett namn och efternamn
    \item En hemadress
    \item En e-postadress såsom namn.efternamn@företag.com
    \item Ett id-kortsnummer
    \item Platsinformation (t.ex. platsfunktionen på en mobiltelefon)
    \item En IP-adress
    \item Kakor
    \item Reklamidentifieraren på din telefon
    \item Uppgifter som innehas av ett sjukhus eller en läkare och som skulle kunna vara en symbol som fungerar som en unik identifikation av en person
\end{itemize}

\subsubsection{Opt-in – aktivt medgivande}
GDPR kräver aktivt medgivande vilket innebär att varje person vars personuppgifter ska
lagras ger aktivt samtycke till detta. Det innebär att de vars personuppgifter vi lagrar
måste ge aktivt samtycke \textit{innan} vi lagrar deras personuppgifter och att hen kan anses
förstå innebörden av datalagringen.

Det räcker inte med att begära ett godkännande för att spara personuppgifterna utan man behöver även förklara varför och hur länge och hur de kommer att behandlas.

Några exempel:
\begin{itemize}
    \item På anmälan till ett evenemang behöver det finnas en obligatorisk kryssruta där den
som anmäler sig godkänner att dess personuppgifter sparas och behandlas av sektionen för att arrangemanget skall kunna genomföras eller för att personen skall kunna
ta del av arrangemanget. Ytterligare skall personen informeras om varför och hur
personuppgifterna behandlas och sparas. Exempel på text:

\textit{\textbf{För att arrangemanget ska gå att planera och genomföra behöver vi spara dina ovan
angivna uppgifter under en begränsad tid. Dina personuppgifter kan komma att delas med
tredje part, t.ex. puffar eller samarrangörer, om det behövs för arrangemanget. Strax efter
arrangemanget kommer uppgifterna att raderas.
}}


    \item På puff-formulär vill spara arbetarnas uppgifter en längre tid än till strax efter arrangemangets slut. Anledningar till detta kan vara att att man vill bjuda arbetarna på ett tackkalas eller be om hjälp vid senare tillfälle. Då kan följande stå:
    
\textit{\textbf{Dina uppgifter kommer att sparas för att vi ska kunna kontakta dig även vid senare tillfälle, t.ex. om vi behöver hjälp vid något annat arrangemang eller på något sätt vill tacka dig för din insats. Dina uppgifter kommer raderas vid slutet av vårt verksamhetsår.}}
        
\end{itemize}

\subsubsection{Tidsbestämt}
Uppgifter får inte sparas längre än nödvändigt. Några exempel:
\begin{itemize}
    \item Anmälningslistor till sittningar skall raderas senast två veckor efter sittningen.
    \item Pufflistor skall raderas senast två veckor efter tackkalaset.
\end{itemize}

\subsubsection{Specificitet}
En personuppgift får bara användas till det som den som äger personuppgiften har godkänt
att den får användas till.

Antag att det finns ett godkännande för att spara e-postadresser för att kunna kontakta
personer. Då är det endast okej för detta syfte och man får då inte använda dessa uppgifter för att skapa en mailinglista till alla som går på data eller dela uppgifterna till tredje part.

\subsubsection{Spara alltid hela namnet}
Fullständigt namn ska alltid sparas tillsammans med andra personuppgifter för en per-
son så att personuppgiftsansvarige har en rimlig möjlighet att t.ex. ta bort en persons
personuppgifter om den vill bli bortglömd.

Det kan vara svårt att få deltagare att fylla i sina namn. Därför rekommenderar vi att
fullständigt namn och smeknamn samlas in som separata fält för att uppmuntra de som
fyller i listan eller formuläret att ge användbar information.

\subsection{Shared Drive}
Shared Drive är som en vanlig My Drive men det är gruppen och inte individerna som äger
innehållet. Varje kommitté och förening får ha en och endast en TeamDrive som de ärver
från tidigare år. För dessa gäller:
\begin{itemize}
    \item Högst upp i mapphierarkin bör det finnas en mapp för varje år, t.ex. “2017” och “2018”.
    \item Alla mappar och dokument döps så att en utomstående förstår vad det handlar om. Olämpliga namn kan vara “Supa satan mapp” och “Lite roligt bös”. Döp dessa istället till “Postom” och “Förslag på arr” så följer ni policyn och gör livet lättare för era efterträdare.
    \item Alla dokument och mappar som innehåller personuppgifter ska ha ett namn med
suffixet “[PU]” (PersonUppgifter). T.ex. mappen “Möten[PU]” eller filen "Kontaktuppgifter sittande[PU]”.
\end{itemize}

Det är inte allt för sällan som man har ett gemensamt arrangemang med andra kommittéer
eller puffar som behöver ta del av ett eller flera dokument.

\begin{itemize}
    \item Om det är puffar som klarar sig med enstaka dokument, dela då endast dessa.
    \item Om det är ett samarr och utomstående/andra kommittéer behöver tillgång till alla filer är det okej att skapa en temporär Shared Drive dit man bjuder in alla arrangörer som behöver tillgång till filerna. Denna Shared Driven ska raderas alternativt ska allt innehåll flyttas efter arrangemanget om inte speciella omständigheter råder. Personuppgiftsansvarige beslutar om det råder speciella omtändigheter. Ovanstående regler gäller även för dessa.
\end{itemize}

\textbf{Tips:} Om det är en hel grupp som ska ha tillgång till en drive räcker det med att gruppen bjuds in för att alla ska få tillgång.

\textbf{OBS!}Tänk på att ni måste ha godkännande att dela till tredje part, se Opt-in ovan. Intresseföreningar och kommittéer på data räknas inte som tredje part eftersom de är en del av datateknologsektionen.

\begin{itemize}
    \item En mapp för varje år. T.ex. “2017”, “2018”.
    \item Det är okej att ha en mapp för alla år tidigare. T.ex. “1337–2016”
\end{itemize}

Vi rekommenderar att man följer ovanstående struktur för att göra det lätt för nya att
snabbt få en överblick över vilka dokument som finns och var de ligger, men om den
strukturen inte fungerar, använd en som fungerar bättre för er.

\subsection{GMail}
Använd enbart e-postadresserna du har genom sektionen till sektionsarbete, det är alltså
både otillåtet och dumt att använda den för privat bruk då du lämnar över adressen till
din efterträdare efter ditt verksamhetsår.

E-post får ej vidarebefordras till privata e-postadresser.

\textbf{Tips:} Sätt en signatur som automatiskt infogas i slutet av dina mejl. Förslag:\\
\textbf{\textit{Vänliga hälsningar,\\
<Förening> <Post> 20xx/20xx\\
Datateknologsektionen Chalmers Studentkår\\
www.dtek.se<http://www.dtek.se/>}}

\subsection{Kalender}
Man får gärna skapa en eller flera kalendrar. Viktigt är att tänka på sätta rätt åtkomsträttigheter och dela med rätt personer. Till exempel kan en bokningskalender som “databussen” eller “Basenpax”, ha följande inställningar:
\begin{itemize}
    \item Tillgänglig för alla: se endast ledig/upptagen (dölj uppgifter).
    \item Delad med: dbus@dtek.se respektive alla@dtek.se. Rätt att göra ändringar och hantera delning.
\end{itemize}

En intern kalender bör endast vara delad med kommittén och en kalender som delas med
utomstående ska inte innehålla personuppgifter som de kan se.

Man kan lägga till hela grupper i en kalender, men då måste varje medlem lägga till
kalendern via en länk som skickas till mejlen. Några bra kalendrar som finns som kan vara
bra för alla datateknologer är:

\begin{itemize}
    \item \textbf{Dtek-kalendern:} \\
    \href{https://calendar.google.com/calendar/embed?src=dtek.se\_0tavt7qtqphv86l4stb0aj3j88\%40group.calendar.google.com\&ctz=Europe\%2FStockholm}{https://calendar.google.com/calendar/embed?src=dtek.se\_0tavt7qtqphv86l4stb0aj3j8\\8\%40group.calendar.google.com\&ctz=Europe\%2FStockholm}

    \item \textbf{Basenpaxningar:}\\
    \href{https://calendar.google.com/calendar/embed?src=dtek.se\_b3sv1v3upmtjquppg10hhe59e0\%40group.calendar.google.com\&ctz=Europe\%2FStockholm}{https://calendar.google.com/calendar/embed?src=dtek.se\_b3sv1v3upmtjquppg10hhe5\\9e0\%40group.calendar.google.com\&ctz=Europe\%2FStockholm}

    \item \textbf{Databussen:}\\
    \href{https://calendar.google.com/calendar/embed?src=dtek.se\_69sdfhe5527mh3u9tk9146imak\%40group.calendar.google.com\&ctz=Europe\%2FStockholm}{https://calendar.google.com/calendar/embed?src=dtek.se\_69sdfhe5527mh3u9tk9146i\\mak\%40group.calendar.google.com\&ctz=Europe\%2FStockholm}

\end{itemize}

\subsection{Handlingsplan vid dataläcka}
Vid minsta misstanke om dataläcka, kontakta omedelbart personuppgiftsansvarig i Styrel-
sen: Styrelsen@dtek.se.


\section{Överträdelse av policy}
För att läsa på om överträdelser se incidenthanteringspolicyn.

%vore bra med ett exempel som vi kan luta oss tillbaka på när vi tar beslut angående ett riktigt straff vi ger ut för att uppskatta hur mycket mildare/värre det någon gjort är.

\end{document}