

\section{Valberednings- och Invalspolicy}

Denna policy beskriver hur valberedningens process ska genomföras samt hur inval till sektionens kommittéer ska ske. Alla medlemmar har rätt att söka alla förtroendeposter inom sektionen.

\subsection{Valberedning} Alla kommittéer har rätt att begära valberedning från styrelsen inför inval av nya kommittémedlemmar. Valberedningens giltighet definieras i sektionens reglemente. Valberedningen har tystnadsplikt gällande diskussioner och detaljer om de sökande. Endast en sammanfattande bedömning av de nominerade ska presenteras vid invalsmötet.

\subsection{Avrådan}
Avrådan utfärdas endast vid enhällighet i valberedningen. Anledningar som krävs för Av-
rådan avgörs av valberedningen, men kan exempelvis vara en av följande:

\begin{itemize}
    \item Involverad i en incident
    \item Upprepat problematiskt beteende
\end{itemize}

Vid bedömd olämplighet ska den sökande informeras vid nomineringarnas publicering,
med en motivering som värnar om samtliga inblandades välmående.
Innan sektionsmötet ska den sökande ta del av en skriftlig motivering och få möjligheten
att diskutera sin Avrådan med en representant från Valberedningen i samråd med SAMO.
På sektionsmötet:

\begin{itemize}
    \item Om den sökande inte ställer sig upp presenteras nomineringarna som vanligt och
Avrådan förblir okänd för övriga.
    \item Om den sökande ställer sig upp, redovisas den skriftliga Avrådan efter nominering-
arna.
\end{itemize}

\subsection{Personinval} Inför inval till en kommitté ska intresserade medlemmar anmäla sitt intresse till sektionsstyrelsen senast 10 dagar före det sektionsmöte där invalet ska ske. Detta gäller även fyllnadsval. Anmälan är dock inte bindande.

En lista med alla de sökande som är intresserade av en förtroendepost ska publiceras i sektionens lokaler senast 9 dagar innan det aktuella mötet.

En sektionsmedlem har fortfarande rätt att söka en post även om denne inte anmält sig i förväg. Vid invalsmötet ska sektionsstyrelsen dock tydligt klargöra vilka av de sökande som har anmält sig inom den angivna tidsfristen.

