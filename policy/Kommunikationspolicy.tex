\section{Kommunikationspolicy}
Arrangemang riktade mot samtliga sektionsmedlemmar samt uttalanden från sektionen och dess kommitter skall omfattas av policyns bestämmelser. Policyns syfte är att försäkra gemene datateknolog tillgång till sektionsviktig information.

\subsection{Annonsering}
Annonsering, eller i folkmun PR, skall följa annonseringskrav för att skapa kontinuitet
och en igenkänningsförsäkran för datateknologer. Annonseringskraven bidrar även till en
rättvis behandling av samtliga datateknologer. Ytterligare gör detta att sektionens olika
kommittéer och dess verksamheter blir mer synliga för gemene teknolog.

\subsubsection{Allmänna råd för annonsering}
Annonsering behöver inte nödvändigtvis göras enbart då arrangemangets alla detaljer är
fastställda. Ett allmänt råd är att annonsera datum och annan information som fastställts
trots eventuell avsaknad av information.

\subsubsection{Medium för annonsering}
Annonsering av arrangemang skall tillgängliggöras via minst två av följande:
\begin{itemize}
    \item Sektionens nyhetsbrev.
    \item Anslagstavlan i Basen.
    \item Evenemang på Facebook med publicering i sektionens officiella Facebook-grupp.
\end{itemize}
Utöver ovannämnda medium uppmuntras arrangörer, i största möjliga utsträckning, att
tillgängliggöra informationen på ytterligare medium som till exempel data-kalendern, kom-
mittesidor och eller andra sociala medier.

\subsubsection{Innehållskrav för annonsering}
All nödvändig information om arrangemanget ska finnas tillgänglig på engelska. Annonser
bör, i största möjliga utsträckning, informera om ifall svenskakunskap är rekommenderat
för deltagare, exempelvis på sittningar ”toastade” på svenska.

Annonser bör innehålla följande information om:
\begin{itemize}
    \item Datum, tid och plats för arrangemanget.
    \item Priser, ifall tillämpningsbart, för att delta vid arrangemanget.
\end{itemize}
I sådana fall där innehållskravet ej kan uppnås som t.ex. om pris eller lokal ej är fastställd
bör det istället framgå hur man förväntas bli informerad.

\subsubsection{Övriga krav för annonsering}
Den som omfattas av policyn bör annonsera arrangemanget i god tid, i regel bör man
annonsera arrangemanget minst sju dagar i förväg. 14 dagar är att anse som mycket god
tid. Detta för att försäkra att informationen hinner nå så många som möjligt. Det kan vara
anmärkningsvärt att sektionsmedlemmar spenderar olika mycket tid på sociala medier och
har olika långt till sektionens lokaler.

\subsection{Kungörelser eller andra underrättelser}
För meddelanden som är av en sådan vikt att deras allmänna spridning på sektionen anses
önskvärd bör kravet för att informationen nås ut till sektionens medlemmar anses vara
högre. Sektionsmöteskallelser omfattas av denna bestämmelse.

Meddelanden som omfattas av denna punkt bör publiceras på alla av följande:
\begin{itemize}
    \item Sektionens nyhetsbrev.
    \item Anslagstavlan i Basen.
    \item Evenemang på Facebook med publicering i sektionens officiella Facebook-grupp.
\end{itemize}
Datum för sektionsmöten bör, i de fall det är tillämplingsbart, tillgängliggöras direkt, även
om den formella kallelsen ej är tillgänglig vid tillfället.

\subsection{Styrelsens kommunikation}
Då styrelsen önskar nå ut till specifika datateknolog(er) så bör e-post användas om ingen
annan känd och effektivare metod existerar. Vid de tillfällen då individens e-post inte är
känd så kommer högskolans tillhandahållna student e-post att användas.

\subsection{Kommunikationskanaler}
Här behandlas sektionen flertal officiella kommunikationskanaler. Ej listade här är kom-
mitteernas egna kommunikationskanaler.

Sektionens officiella kommunikationskanaler:
\begin{itemize}
    \item Sektionens Facebook-grupp,\\
Datateknologsektionen Chalmers,\\
www.fb.com/groups/cth.dtek
    \item Sektionens Nyhetsbrev på e-post,\\
Veckobrevet,\\
www.dtek.se/newsletter
    \item Anslagstavla (övre korridor),\\
Sektionens Anslagstavla,\\
I Basen-korridoren, tavlan höger om Undercentral-dörren
    
    \item Anslagstavla (nedre korridor),\\
Sektionens Anslagstavla,\\
I Basen-korridoren, tavlan efter korridorstrapporna
    \item Sektionens Twitter-sida,\\
Datateknologsektionen Chalmers,\\
www.twitter.com/datachalmers
    \item Sektionens Facebook-sida,\\
Datateknologsektionen,\\
www.facebook.com/dtekchalmers

\end{itemize}
