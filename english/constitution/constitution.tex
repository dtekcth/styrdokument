% LaTeX-inställningar%%%%%%%%%%%%%%%%%%%%%%%%%%%%%%%%%%%%%%%%%%%%%%%%%%
% Kräver dtek-tex (github.com/dtekcth/dtek-tex)
% KOMPILERAS MED xelatex!
\documentclass[a4paper]{dtek}
\setcounter{secnumdepth}{5}
\title{Constitution}

\newcommand{\sdocsae}{Student Division of Computer Science and Engineering~}

%FYLL I VID ÄNDRINGAR%%%%%%%%%%%%%%%%%%%%%%%%%%%%%%%%%%%%%%%%%%%%%%%%%%
\newcommand{\updated}{2019-02-07} %Insert

%Dokumentstart%%%%%%%%%%%%%%%%%%%%%%%%%%%%%%%%%%%%%%%%%%%%%%%%%%%%%%%%%
\begin{document}
\makeheadfoot

%Titlepage%%%%%%%%%%%%%%%%%%%%%%%%%%%%%%%%%%%%%%%%%%%%%%%%%%%%%%%%%%%%%
\vspace*{\fill}
\begin{center}
{\Huge \textbf{Constitution of the \sdocsae}}
\par\bigskip
\includegraphics[width=300pt]{dteklogo.pdf}
\par\bigskip
{\LARGE Chalmers, Gothenburg}
\end{center}
\vspace*{\fill}
\begin{center}
{\LARGE Updated: \updated}
\end{center}
\vspace*{\fill}


\newpage
\setcounter{tocdepth}{1}
\tableofcontents
\newpage


%Innan vi börjar...%%%%%%%%%%%%%%%%%%%%%%%%%%%%%%%%%%%%%%%%%%%%%
\setcounter{section}{-1}
\section{Before we begin\dots}
This is a translation of the Swedish version of the document. In case there's a difference between the English and Swedish versions, the Swedish version prevails.

This translation was written by Tove and Hugo, and is guided by the dictionary written by Talhenspresidiet/Speakers' Council '18

The English terms for meeting formalities in this document are mostly taken from ``Robert's Rules of Order, Newly Revised'', a book which includes the rules most commonly used in the U.S. when conducting meetings.

\newpage

%Allmänt%%%%%%%%%%%%%%%%%%%%%%%%%%%%%%%%%%%%%%%%%%%%%%%%%%%%%%%%%
\section{General}
\subsection{Purpose} 
\subsubsection{}
The \sdocsae at Chalmers University of Technology, hereafter referred to as ``The student division'', is a nonprofit organization consisting of students at the Computer Science and Engineering program at Chalmers University of Technology.
\subsubsection{}
The Student Division's purpose is to act for unity among its members and to look after their common interests in educational and social issues. 
\subsubsection{}
The student division is independent of trade unions, party politics and religion.

\subsection{Members}
Anyone who is enrolled in the Computer Science and Engineering program at Chalmers and has paid the student division fee are members. In addition to that, anyone who is both a former student the the Computer Science and Engineering program at Chalmers and who have paid an administrative fee are also members. The student division may also have honorary members.  

\subsection{Fiscal year}
The student division's fiscal year runs from the 1st of May to the 30th of April.
\newpage

%Medlemmar%%%%%%%%%%%%%%%%%%%%%%%%%%%%%%%%%%%%%%%%%%%%%%%%%%%%%%%%%
\section{Members}
\subsection{Rights}
\subsubsection{}
During division meetings, all members have the right to attend, to speak, to propose matters and to vote.

\subsubsection{}
Only members are eligible to be elected to a position within the student division. The lay-auditors are exempted from the previous rule, where "position" means selected by the division meeting or the division board.

\subsubsection{}
Members have the right to access the minutes of the meeting and the student division's other documents.


\subsection{Requirements}
Members are required to comply with the regulations of the student division. 

\subsection{Rights of honorary members}
Honorary members have the right to attend and to speak at division meetings.

%Organisation%%%%%%%%%%%%%%%%%%%%%%%%%%%%%%%%%%%%%%%%%%%%%%%%%%%%%%%%%
\section{Organization}
\subsection{Exercise of operations}
The operations of the student division are exercised as described in this constitution with associated by-laws and financial regulations through: 

\begin{enumerate}
    \item Division meetings
    \item The board of the division 
    \item The division nomination committee 
    \item The lay-auditors
    \item The educational committee of the division (DNS)
    \item Other division comittees
    \item Other interest comittees
\end{enumerate}

\subsection{Responsibilities}
\subsubsection{}
The division meeting is the student division's highest decision-making body. 
The board of the division acts for the division meeting between division meetings. 

\subsubsection{}
The division meeting has at its disposal the nomination committee, the lay-auditors, the division committees, the interest committees, the educational committee and the board of the division. 

\newpage

%Sektionsmötet%%%%%%%%%%%%%%%%%%%%%%%%%%%%%%%%%%%%%%%%%%%%%%%%%%%%%%%%%
\section{The division meeting}
\subsection{Powers}
The division meeting is the student division's highest decision-making body.
\subsection{Meetings}
There is to be four ordinary division meetings annually, one per study period. Furthermore there can be extraordinary division meetings. 

\subsection{Calling a meeting}
\subsubsection{}
The division meeting is held by request of the speakers' council or the board of the division. 

\subsubsection{}
The right to demand that a division meeting be called is held by any member of the division board, the inspector, the student union inspector, the student union board, the division auditors or at least 25 members of the division. Such a request shall be presented to the speakers' council or the division board. A meeting called like so is to be held within 10 study days of the request. 

\subsubsection{}
\label{sec:sektionsmote_utlysande}
Division meetings shall be proclaimed five study days in advance by summons in accordance with the by-laws. Received proposals are to be posted at least three study days ahead of the meeting. 


\subsection{Requirements}
\subsubsection{}
At the latest, the following shall be handled by the division meeting one day before the start of a new term: 

\begin{itemize}
    \item Allocation of the division's and division committees' resources.
    \item Election of the board of the division. 
    \item Election of lay-auditors.
    \item Election of the inspector if needed. 
\end{itemize}


\subsubsection{}
At the latest, the following shall be handled by the division meeting one day before the start of the fiscal year: 
\begin{itemize}
    \item Membership fee for the following two semesters.
    \item Deciding on a preliminary budget for the coming fiscal year. 
\end{itemize}


\subsubsection{}
At the latest, the following shall be handled by the division meeting six months after the start of the fiscal year: 

\begin{itemize}
    \item The division's and the division committees' yearly report as well as the audit reports for the previous fiscal year. 
    \item Resolution of discharge.
    \item Deciding on a final budget for the current fiscal year.
\end{itemize}


\subsection{Quorum}
\subsubsection{}
The division meeting has a quorum if it is called in accordance with chapter ~\ref{sec:sektionsmote_utlysande} of the constitution. 

\subsubsection{}
Additionally, if fewer than 40 members of the division are present at the division meeting when a decision is to me made, it can only pass if no one moves to postpone. The same applies to matters that have not been announced three study days in advance of the meeting. 

\subsection{Proposals}
Members that wish to include a matter on the agenda shall send it in written form to the board of the division at least five study days in advance of the meeting. 

\subsection{Appeal}
Decisions made by the division meeting that are not in accordance with the student union constitution or the student division's constitution, by-laws, financial regulations or other policy can be removed by the student union council. Such a decision shall be appealed if it is demanded by a member of the student union if it concerns the student union constitution or by a division member if it concerns the division constitution, by-laws, financial regulations or policy. 


\subsection{Voting}

\subsubsection{}
Voting by proxy is not allowed. 

\subsubsection{}
Voting is open unless anyone moves that the vote be conducted by ballot.

\subsubsection{}
In the case of equal votes the chairman of the meeting breaks the tie. In the case of elections, the winner is chosen by drawing lots.

\subsubsection{}
In the case of a vote concerning several proposals, the voting procedure shall be determined before voting commences.

\subsubsection{}
All matters are handled with simple majority if nothing else is specified in the constitution. Votes to abstain are not counted. 

\subsection{Right to attend and to speak}
The right to attend and speak at the student division meeting is held by members of the division, honorary members, student union board members, the inspector, lay-auditors and co-opted non-members. 

\subsection{Right to propose matters}
The right to propose matters before the meeting is held by members of the division, the inspector and co-opted non-members. 

\subsection{Right to vote}
The right to vote is held by members of the division. 

\subsection{Minutes}
The minutes of the meeting shall be adjusted by two adjusters chosen by the meeting. Adjusted minutes shall be posted at the latest ten study days after the meeting. 

\newpage

%Valberedning%%%%%%%%%%%%%%%%%%%%%%%%%%%%%%%%%%%%%%%%%%%%%%%%%%%%%%%%%
\section{Nomination committee}
\subsection{Composition}
\subsubsection{}
The convener for the nomination commitee is chosen by the division board. 

\subsubsection{}
Representatives in the nomination commitee are prescribed in the by-laws. 

\subsection{Responsibilities}
The nomination commitee is responsible for nominating to elected positions within the division. 

\subsection{Proclamation}
The nomination committee's nominations shall be posted at least five study days ahead of the division meeting. 

\subsection{Free nomination}
Free nomination is allowed for all positions with the exception of chairman and treasurer of the division board. For these positions only those who have announced interest to the board at least 24 hours before the meeting can be nominated. 

\newpage

%Sektionsstyrelsen%%%%%%%%%%%%%%%%%%%%%%%%%%%%%%%%%%%%%%%%%%%%%%%%%%%%%
\section{Board of the division}
\subsection{Authorities}
The student division board acts in accordance with this constitution, the by-laws, financial regulations and decisions made by the division meeting, as the executive management for the division. 


\subsection{Composition}
The members of the board of the division are: 
\begin{itemize}
    \item Chairman
    \item Deputy chairman
    \item Treasurer
    \item Secretary
    \item Student safety and welfare representative (SAMO)
    \item other members as determined in the by-laws
\end{itemize}

The chairman and the treasurer shall not be minors. 

\subsection{Rights}
The division board has the right to use the division's name and emblems in accordance with the student union policies. 

\subsection{Responsibilities}
The board of the division is responsible to the division meeting for the operation and finances of the division. 

\subsection{Authorized signatory}
The chairman and treasurer of the division board sign the division separately. 
%mycket oklar


\subsection{Board meeting}
The board of the division convenes at least three times per study period. 

\subsection{Proclamation}
\subsubsection{}
The board convenes at the summons of the chairman or deputy chairman of the board.

\subsubsection{}
Members of the board hold the right to demand that deputy chairman summons the board. Such a meeting shall be held within five study days. 

\subsection{Quorum}
50\% of the members of the board constitutes a quorum. Addtionally, the chairman or deputy chairman must be present. 

\subsection{Appeal}
Decisions made by the board that are not in accordance with the student union constitution or the student division's constitution, by-laws, financial regulations or other policy can be removed by the student union council. Such a decision shall be appealed if it is demanded by a member of the student union if it concerns the student union constitution or by a division member if it concerns the division constitution, by-laws, financial regulations or policy. 

\subsection{Minutes}
Minutes shall be taken at board meetings. They shall be adjusted by two members of the board and posted at the division notice board at the latest five study days after the meeting. 


\subsection{Vote of no confidence}
\subsubsection{}
A vote of no confidence in the board of the division has to be proclaimed three study days before the division meeting. At least 2/3 of attending members have to vote in favour for the vote of no confidence to pass.

\subsubsection{}
After a successful motion of no confidence, an intermission board and a new nomination committee shall be elected. The intermission board shall proclaim the summons for an extraordinary division meeting where a new board shall be elected. This meeting is to be held during the semester within 15 study days from the meeting where the intermission board was elected.

\subsubsection{}
The intermission board takes over all authorities and responsibilities of the  board of the division until a new board is elected. 

\newpage

%Studienämnden%%%%%%%%%%%%%%%%%%%%%%%%%%%%%%%%%%%%%%%%%%%%%%%%%%%%%%%%%
\section{The educational committee of the division (DNS)}
\subsection{Function}
\subsubsection{Reponsibilities}
The Computer Engineering Student Division Educational Committee, DNS, is responsible for monitoring educational issues within the division. 

\subsubsection{Composition}
The \sdocsae Educational Committee consists of a chairman, a deputy chairman and up to four other members. All members are elected by the division meeting. 

\subsubsection{Course evaluation}
DNS are responsible for following up on and participating in the conducting of course evaluations. 

\subsection{Rights}
The \sdocsae Educational Committee has the right to use the division's name and emblems in accordance with the student union policies. 

\subsection{Obligations}
The \sdocsae Student Division Educational Committee is obliged to act in accordance with the provisions and decisions of the division. 

\newpage

%Sektionsföreningar%%%%%%%%%%%%%%%%%%%%%%%%%%%%%%%%%%%%%%%%%%%%%%%%%%%%
\section{Division committees}
\subsection{Definition}
\subsubsection{}
A division committee must have a chairman, a treasurer and a number of other members determined by the by-laws. 

\subsubsection{}
Positions are appointed by the division meeting with advice from the nomination commitee. 

\subsubsection{}
Division committees shall act in the best interest of the division and have a function determined by the by-laws. 

\subsubsection{}
Chairmen and treasurers of the division committees must not be minors.

\subsection{Rights}
The division committees has the right to use the division's name and emblems in accordance with the student union policies.

\subsection{Obligations}
The division committees are obligated to act in accordance with the constitution, by-laws, financial regulations and decisions of the division. 


\subsection{Finances}
\subsubsection{}
The chairman and treasurer of the division committee sign the committee separately. 

% Också mycket oklar. Se styret


\subsubsection{}
The operation and finances of the division committees is audited by the division lay-auditors. 

\subsection{Record}
The student division committees are recorded in the by-laws.
\newpage

%Intresseföreningar%%%%%%%%%%%%%%%%%%%%%%%%%%%%%%%%%%%%%%%%%%%%%%%%%%%%
\section{Interest committees}
\subsection{Definition}
\subsubsection{}
The interest committees have a chairman and a number of other interested members. 

\subsubsection{}
The chairman is elected as determined in the regulations. 

\subsubsection{}
Interest committees shall act in the best interest of the division and have a function determined by the by-laws. 
\subsubsection{}
Chairmen of interest committees must not be minors.
\subsection{Rights}
The interest committees has the right to use the division's name and emblems in accordance with the student union policies.
%%%%
\subsection{Obligations}
The interest committees are obligated to act in accordance with the constitution, by-laws, financial regulations and decisions of the division. 

\subsection{Record}
The interest committees are specified in the by-laws.
\newpage

%Hedersmedlemmar%%%%%%%%%%%%%%%%%%%%%%%%%%%%%%%%%%%%%%%%%%%%%%%%%%%%%%%
\section{Honorary members}
\subsection{Basic requirement}
A person that has particularly furthered the divisions interests and goals can be accepted as an honorary member. 


\subsection{Nomination and summons}
Nomination of honorary members are to be given in writing to the board of the division at the latest five study days before the division meeting, and be signed by at least 25 members of the division. The decision to accept the honorary member is made at the next division meeting and is only valid if 2/3 of the voting members vote for. If the summoned person accepts the summons he or she is officially an honorary member. 


\newpage

%Skyddshelgon och sektionsfärger%%%%%%%%%%%%%%%%%%%%%%%%%%%%%%%%%%%%%%%
\section{Patron saint and colour}
\subsection{Patron saint}
The patron saint of the student division is Woody Woodpecker. 

\subsection{Colour}
The student division colour is orange. 

\newpage

%Protokoll och anslagning%%%%%%%%%%%%%%%%%%%%%%%%%%%%%%%%%%%%%%%%%%%%%%
\section{Minutes and posting}
\subsection{General}
Minutes taken within the division organization shall include notes of the nature of the matters as well as all raised and not withdrawn proposals, decisions, special proposals and reservations. 


\subsection{Proclaiming}
Messages and decisions are correctly proclaimed when posted on the official notice board of the division. 

\newpage

%Revision och ansvarsfrihet%%%%%%%%%%%%%%%%%%%%%%%%%%%%%%%%%%%%%%%%%%%%
\section{Audit and discharge}
\subsection{Auditors}
\subsubsection{}
The division meeting appoints two lay-auditors tasked with auditing the division's operations and finances during the fiscal year. 

\subsubsection{}
Division auditors may not hold any other position within the division during their fiscal year. 

\subsubsection{}
Bookkeeping and other documents shall be made available to the lay-auditors at least 15 study days before the division meeting. 

\subsection{Responsibilities}
\subsubsection{}
It is the lay-auditors' responsibility to post audit reports on the official notice board at least three study days before the ordinary division meeting. 

\subsubsection{}
Audit reports shall contain a statement concerning discharge of affected parties. 

\subsection{Discharge}
\subsubsection{}
Discharge is granted to affected parties if the division meeting decides so. 

\subsubsection{}
If a person elected to a position within the division retires from their post before the end of their term, an audit shall be undertaken. 
%oklart
% Skulle förtroendevald på teknologsektionen med ekonomiskt ansvar avgå före mandatperiodens slut, skall revision företagas.
\newpage

%Avgifter%%%%%%%%%%%%%%%%%%%%%%%%%%%%%%%%%%%%%%%%%%%%%%%%%%%%%%%%%%%%%%
\section{Fees}
\subsection{Division fee}
Every member of the student division shall pay the decided division fee. 
\subsection{Administrative fee}
A member who is not a student shall pay the decided administrative fee. 

\newpage

%Teknologsektionens upplösning%%%%%%%%%%%%%%%%%%%%%%%%%%%%%%%%%%%%%%%%%
\section{Dissolution of the student division}
\subsection{Decision of dissolution}
The student division is dissolved through decision at two subsequent division meetings separated by at least 15 study days, with at least 60 or all members of the division present. For the decision of dissolution to be valid at least three quarters of the voting members have to be in favor. 


\subsection{Resources and relaunch}
If the division meeting decides to dissolve the division, all assets and debts as stated in the prepared balance sheet will be passed on to the Chalmers Student Union until a new association or student division that represents students in the program of Computer Science and Engineering at Chalmers is founded.

\newpage

%Ändrings- och tolkningsfrågor%%%%%%%%%%%%%%%%%%%%%%%%%%%%%%%%%%%%%%%%%
\section{Amendments and questions of interpretation}
\subsection{Changing the constitution}
\subsubsection{}
Changing to this constitution can only be made by the division meeting. To be valid, the change has to be adopted with 2/3 of the voting in favour on two consecutive division meetings, where at least one of them is an ordinary division meeting, with at least 10 study days separating them. 

\subsubsection{}
Amendment or addition to this constitution must be approved by the student union board. 

\subsection{Amendment of the by-laws}
Amendment or addition to the by-laws or financial regulations can only be made by the division meeting. To be valid the amendment must be adopted with 2/3 of the voting in favour. 

\subsection{Question of interpretation}
\subsubsection{}
If there is a question of interpretation regarding this constitution they will be interpreted by the inspector. If there is no inspector the matter will be resolved by the student union inspector. 

\subsubsection{}
In the matter of interpreting the by-laws and financial regulations the board's interpretation is valid until the matter is resolved by the division meeting. 

\newpage

%Inspektor%%%%%%%%%%%%%%%%%%%%%%%%%%%%%%%%%%%%%%%%%%%%%%%%%%%%%%%%%%%%%
\section{Inspector}
\subsection{General}
The inspector shall attend to and support the division's operations. The inspector shall therefore be held up to date on the division's operations. The inspector has the right to access any minutes and other documents the division may have.
\subsection{Election}
The inspector is elected by the division meeting and holds the post for two calendar years. 
\newpage

%Hedersbetygelser%%%%%%%%%%%%%%%%%%%%%%%%%%%%%%%%%%%%%%%%%%%%%%%%%%%%%%
\section{Honours}
\subsection{General}
The student division can give out bar mirrors as thanks or as a mark of honour. 

\subsection{Criteria}
To be regarded deserving of a bar mirror one of the following criteria should be met: 

\begin{itemize}
    \item having done the division a significant service
    \item having done the division a significant disservice
    \item be a monarch and turn an age that is divisible by five
\end{itemize}

\subsection{Obtainment}
It is the board of the division's responsibility to make sure the required amount of bar mirrors are available. 
\end{document}