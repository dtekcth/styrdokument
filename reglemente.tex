%LaTeX-inställningar%%%%%%%%%%%%%%%%%%%%%%%%%%%%%%%%%%%%%%%%%%%%%%%%%%%
% Kompliera med pdflatex!!!
\documentclass[a4paper, 10pt]{article}

\usepackage{hyperref}
\usepackage[utf8]{inputenc}
\usepackage[T1]{fontenc}
\usepackage[swedish]{babel}
\usepackage{graphicx}
%\usepackage{ae} %För riktiga fonts?
\usepackage{fancyhdr}
\usepackage{lastpage}
\topmargin -20.0pt
\headheight 56.0pt
\setcounter{secnumdepth}{5}

%FYLL I VID ÄNDRINGAR%%%%%%%%%%%%%%%%%%%%%%%%%%%%%%%%%%%%%%%%%%%%%%%%%%
\newcommand{\updated}{2013-03-19} %Insert




%Dokumentstart%%%%%%%%%%%%%%%%%%%%%%%%%%%%%%%%%%%%%%%%%%%%%%%%%%%%%%%%%
\begin{document}
\pagestyle{fancy}

%Header%%%%%%%%%%%%%%%%%%%%%%%%%%%%%%%%%%%%%%%%%%%%%%%%%%%%%%%%%%%%%%%%
\renewcommand{\headrule}{\vbox to 0pt{\hfill\hbox to 292pt{\hrulefill}}}
\lhead{
\raisebox{-25pt}[0pt][10pt]{\includegraphics[width=60pt]{Datalogo.pdf}}
\parbox[b]{200pt}{
\textbf{Datateknologsektionen}\\
Chalmers studentkår\\
Reglemente}}
\rhead{ \flushright
Sidan \thepage\ av \pageref{LastPage}\\
Uppdaterad \updated}

%Footer%%%%%%%%%%%%%%%%%%%%%%%%%%%%%%%%%%%%%%%%%%%%%%%%%%%%%%%%%%%%%%%%
\renewcommand{\footrulewidth}{\headrulewidth}
\lfoot{\flushleft Datateknologsektionen\\
    Rännvägen 8\\
    412 58 Göteborg}
\cfoot{}
\rfoot{ \flushright styret@dtek.se\\
www.dtek.se}
\newpage

%Titlepage%%%%%%%%%%%%%%%%%%%%%%%%%%%%%%%%%%%%%%%%%%%%%%%%%%%%%%%%%%%%%
\vspace*{\fill}
\begin{center}
{\Huge \textbf{Reglemente för Datateknologsektionen}}\\
\includegraphics[width=300pt]{Datalogo.pdf}
\\{\LARGE Chalmers, Göteborg}
\end{center}
\vspace*{\fill}
\begin{center}
{\LARGE Uppdaterad: \updated}
\end{center}
\vspace*{\fill}


\newpage
\setcounter{tocdepth}{1}
\tableofcontents
\newpage

%Sektionsmötet%%%%%%%%%%%%%%%%%%%%%%%%%%%%%%%%%%%%%%%%%%%%%%%%%%%%%%%%%
\section{Sektionsmötet}

\subsection{Kallelse} 
Kallelse till sektionsmöte består av förslag till dagordning. Förslag till dagordning skall anslås på teknologsektionens officiella anslagstavla. \\\\
Teknologsektionens officiella anslagstavla skall vara placerad i Basen. 
\subsubsection{Förslag till dagordning}
Förslag till dagordningen skall innehålla:
\begin{itemize}  
  \item Datum, tid och plats för mötet 
  \item Mötets öppnande 
  \item Val av mötesordförande 
  \item Val av mötessekreterare 
  \item Val av två justeringsmän tillika rösträknare 
  \item Föregående mötes protokoll. 
  \item Mötets behöriga utlysande 
  \item Fastställande av mötesordning 
  \item Adjungeringar 
  \item Meddelanden 
  \item Eventuella verksamhetsrapporter från sektionsföreningar 
  \item Eventuella års- och revisionsberättelser 
  \item Eventuella personval 
  \item Propositioner 
  \item Motioner 
  \item Övriga frågor 
\end{itemize}
\subsection{Mötesplats} 
Sektionsmötet skall hållas på Chalmersområdet. 
\subsection{Mötesordning}
\subsubsection{Begärande av ordet}
Ordet begärs genom handuppräckning och delas i tur och ordning ut av mötesordförande. 
\subsubsection{Replik}
Om ett anförande berör en speciell person, har denne rätt till replik om högst en minut. Replik skall hållas i direkt anslutning till anförandet. En kontrareplik om högst en minut kan beviljas. Kontra-kontra-replik beviljas ej. 
\subsubsection{Ordningsfråga}
Debatt i ordningsfråga bryter debatt i sakfråga och skall avgöras innan debatt i sakfråga fortsätter. 
\subsubsection{Streck i debatten}
Streck i debatten behandlas som ordningsfråga. Bifalls frågan om streck i debatten skall mötesordföranden justera talarlista. Därefter får endast de som står på listan yttra sig i frågan och inga nya yrkanden i sakfrågan får framställas. 
Upphävande av streck i debatten behandlas även det som ordningsfråga. 
\subsubsection{Yrkande}
Yrkande framställs både muntligt och skriftligt till mötesordförande. 
\subsubsection{Reservation}
Reservation mot beslut av sektionsmötet skall anmälas muntligt i omedelbar anslutning till beslutet och skriftligen senast 24 timmar efter mötet. 
\subsubsection{Ajournering}
Behandlas som ordningsfråga. Bifalls yrkandet om ajournering skall tidslängden av ajourneringen fastställas. 
\subsubsection{Motion} 
Motion som är upptagen på dagordningen måste lyftas och föredragas av motionären eller någon annan på mötet med förslagsrätt, annars faller motionen. Vidare skall styrelsen lämna sitt utlåtande om motionen och därefter följer allmän debatt. 
\newpage
%Sektionsstyrelsen%%%%%%%%%%%%%%%%%%%%%%%%%%%%%%%%%%%%%%%%%%%%%%%%%%%%%
\section{Sektionsstyrelsen}
\subsection{Ansvarsområden}
\subsubsection{Det åligger sektionsstyrelsen att:}
\begin{itemize}
  \item verka för sammanhållningen mellan sektionsmedlemmarna och verka för deras gemensamma intressen. 
  \item leda sektionens arbete 
  \item övervaka genomförandet av sektionsmötesbeslut och se till att de verkställs 
  \item framlägga budget med förslag på sektionsavgift till sektionsmötet 
  \item framlägga preliminär verksamhetsplan vid sista ordinarie vårmötet 
  \item lämna förslag på representanter till sektionens valberedning 
  \item utse representant till SSD\&IT 
  \item utse representanter till kårledningsutskottet och andra organ 
\end{itemize}
\subsubsection{Det åligger sektionsordförande att:} 
\begin{itemize}
  \item tillse att sektionens beslut verkställs 
  \item föra sektionens talan då något annat ej stadgats eller beslutats 
  \item teckna sektionens firma 
  \item leda och övervaka arbetet inom sektionsstyrelsen 
\end{itemize}
\subsubsection{Det åligger sektionens vice ordförande att:}
\begin{itemize}
  \item biträda ordföranden i dennes värv 
  \item i ordförandens frånvaro överta dennes åligganden 
  \item kontrollera så att arbetet sker i enighet med sektionens bestämmelser 
\end{itemize}
\subsubsection{Det åligger styrelsens kassör att:}
\begin{itemize}
  \item följa sina skyldigheter enligt det ekonomiska reglementet. 
\end{itemize}
\subsubsection{Det åligger styrelsens sekreterare att:}
\begin{itemize}
  \item föra protokoll vid styrelsens möten och tillse att protokoll från såväl styrelse- som sektionsmöten anslås
  \item tillse att material som inkommer till sektionen anslås eller på annat sätt förmedlas till berörda parter.
\end{itemize}
\subsubsection{Det åligger Studienämndens ordförande samt övriga föreningsordförande att:}
\begin{itemize}
  \item bistå styrelsen med information 
  \item aktivt deltaga i beslutsprocessen 
  \item redogöra för sin förenings löpande verksamhet vid styrelsens möten 
\end{itemize}
\subsection{Insyn} 
Sektionsstyrelsen har full insyn i teknologsektionen alla organ och äger rätt att deltaga i deras möten med yttranderätt. 
\subsection{Medlemmar} 
Styrelsen består, utöver de i stadgarna uppräknade, även utav följande föreningars ordförande. 
\begin{itemize}
  \item DRUST 
  \item DAG 
  \item Delta 
  \item D6 
  \item DnollK 
  \item SND 
\end{itemize}
\subsection{Suppleant} 
Då övrig medlem ej kan närvara vid möte har denne rätt att utse en suppleant ur samma förening. 
\subsection{Omröstning} 
\begin{itemize}
  \item Röstning med fullmakt får ej ske. 
  \item Omröstning skall ske öppet. 
  \item Vid lika utfall äger mötesordförande utslagsröst. 
  \item Då flera förslags ställs mot varandra skall röstningsförfarandet fastslås innan omröstning påbörjas.
\end{itemize}
\newpage
%Studienämnden%%%%%%%%%%%%%%%%%%%%%%%%%%%%%%%%%%%%%%%%%%%%%%%%%%%%%%%%%
\section{Studienämnden}
\subsection{Medlemmar}
Studienämnden, SND, består av de i stadgarna uppräknade medlemmarna.
\subsection{Ansvar}
\subsubsection{Det åligger SND att:}
\begin{itemize}
  \item ombesörja att kurser utvärderas
  \item sammanträda så ofta som rådande former för kursutvärdering och deltagande i andra studierelaterade frågor fordrar.
\end{itemize}
\subsubsection{Det åligger SNDs ordförande att:}
\begin{itemize}
  \item i studiefrågor representera datateknologsektionen och föra dess talan
  \item inför sektionen svara för att teknologernas intressen i studiefrågor bevakas.
\end{itemize}
\subsubsection{Det åligger SNDs vice ordförande att:}
\begin{itemize}
  \item understödja ordföranden då denne tillfälligtvis är oförmögen att sköta sina ålägganden.
\end{itemize}
\newpage
%Sektionsföreningar%%%%%%%%%%%%%%%%%%%%%%%%%%%%%%%%%%%%%%%%%%%%%%%%%%%%
\section{Sektionsföreningar}
Teknologsektionen har följande sektionsföreningar: 
\begin{itemize}
  \item DRUST 
  \item DAG 
  \item Delta 
  \item D6 
  \item DnollK
  \item D-foto
  \item D-Lat
  \item DBus
\end{itemize}
\subsection{Uppdrag, val och mandatperiod.}
\subsubsection{DRUST} 
\paragraph{Uppdrag\\}
DRUST har i uppdrag att ansvara för teknologsektionens lokaler.
\paragraph{Val\\}
Ordförande, kassör och 4 övriga medlemmar väljs av sektionsmötet. 
\paragraph{Mandatperiod\\}
Mandatperioden är densamma som verksamhetsåret. 
\subsubsection{DAG}
\paragraph{Uppdrag\\} 
DAG har i uppdrag att verka för samverkan mellan datateknologsektionen och arbetsmarknaden. 
\paragraph{Val\\}
Ordförande, kassör och 4-5 övriga medlemmar väljs av sektionsmötet.
\paragraph{Mandatperiod\\}
Mandatperioden är densamma som verksamhetsåret
\subsubsection{Delta}
\paragraph{Uppdrag\\}
Delta har i uppdrag att verka för sektionsfrämjande aktiviteter. 
\paragraph{Val\\}
Ordförande, kassör och 4 övriga medlemmar väljs av sektionsmötet. 
\paragraph{Mandatperiod\\}
Mandatperioden är densamma som verksamhetsåret. 
\subsubsection{D6}
\paragraph{Uppdrag\\}
D6 har i uppdrag att arrangera fester. 
\paragraph{Val\\}
Ordförande, kassör och 4-6 övriga medlemmar väljs av sektionsmötet. 
\paragraph{Mandatperiod\\}
Mandatperioden är densamma som verksamhetsåret. 
\subsubsection{DnollK}
\paragraph{Uppdrag\\}
DnollK har i uppdrag att sköta nollningen. 
\paragraph{Val\\}
Ordförande, kassör och 4-5 övriga medlemmar väljs av sektionsmötet. 
\paragraph{Mandatperiod\\}
Mandatperioden är 1:a januari – 31:a december.
\subsubsection{D-foto}
\paragraph{Uppdrag\\}
D-foto har i uppdrag att i bild dokumentera teknologsektionens verksamhet.
\paragraph{Val\\}
Ordförande, kassör och 0-4 övriga medlemmar väljs av sektionsmötet.
\paragraph{Mandatperiod\\}
Mandatperioden är densamma som verksamhetsåret.
\subsubsection{D-Lat}
\paragraph{Uppdrag\\} 
D-Lat har i uppdrag att förvalta sektionens alkoholtillstånd och allt som har samröre med detta.
\paragraph{Val\\}
Till ordförande väljs automatiskt sektionens ordförande. Till kassör väljs automatiskt sektionens kassör.
\paragraph{Mandatperiod\\}
Mandatperioden är densamma som verksamhetsåret.
\subsubsection{DBus}
\paragraph{Uppdrag\\}
DBus har i uppdrag att förvalta sektionens fordon och allt som har samröre med detta.
\paragraph{Val\\}
Bilansvarig och Vice Bilansvarig väljs av Sektionsmötet. Personerna skall inneha giltigt B-körkort. Om sådan persoer ej finnes åläggs Styret att finna och tillsätta lämpliga personer, dock ej mot dessas vilja. Till kassör väljs automatiskt sektionens kassör.
\paragraph{Mandatperiod\\}
Mandatperioden är densamma som verksamhetsåret.
\newpage
%Intresseföreningar%%%%%%%%%%%%%%%%%%%%%%%%%%%%%%%%%%%%%%%%%%%%%%%%%%%%
\section{Intresseföreningar}
Sektionen har följande intresseföreningar:
\begin{itemize}
  \item D-lirium 
  \item iDrott 
  \item DAF 
  \item DDD
  \item dHack
  \item Ståthållarämbetet
  \item Datas Ludologer 
  \item Datas vargbröder
\end{itemize}
\subsection{Uppdrag, val och mandatperiod}
\subsubsection{D-lirium}
\paragraph{Uppdrag\\}
Skall roa och kritiskt granska sektionen, kåren och Chalmers. 
\paragraph{Val\\}
Fyra redaktörer väljs av sektionsmötet. Chefredaktör, tillika ansvarig utgivare utses internt av de valda redaktörerna som meddelas sektionsstyrelsen. 
\paragraph{Mandatperiod\\}
Mandatperioden är samma som verksamhetsåret. 
\subsubsection{iDrott}
\paragraph{Uppdrag\\}
Skall främja datateknologens idrottsliga utövande. 
\paragraph{Val\\}
Ordförande och 0-4 övriga medhjälpare väljs av sektionsmötet. 
\paragraph{Mandatperiod\\}
Mandatperioden är samma som verksamhetsåret. 
\subsubsection{DAF}
\paragraph{Uppdrag\\}
Ansvariga för akvariets drift och skötsel. 
\paragraph{Val\\}
Inget val. Ordförande väljs internt. 
\subsubsection{DDD}
\paragraph{Uppdrag\\}
Öka sammanhållningen bland tjejer på datateknologsektionen. 
\paragraph{Val\\}
Inget val. Ordförande väljs internt.
\subsubsection{dHack}
\paragraph{Uppdrag\\}
Skall driva teknologsektionens IT-tjänster och datorsystem, samt
främja hackerandan på sektionen.
\paragraph{Val\\}
En Webbmästare och 0 till 3 övriga väljs av sektionsmötet
\paragraph{Mandatperiod\\}
Mandatperioden är samma som verksamhetsåret
\subsubsection{Ståthållarämbetet}
\paragraph{Uppdrag\\}
Se efter d08gong med tillbehör samt sektionens fana och representera sektionen på traditionsenliga arrangemang
\paragraph{Val\\}
Fanbärare, cermonimästare och 0-2 vapendragare väljs av sektionsmötet.
\paragraph{Mandatperiod\\}
Mandatperioden är samma som verksamhetsåret. 
\subsubsection{Datas Ludologer}
\paragraph{Uppdrag\\}
Bevarande av datateknologens spelintressen, digitala såväl som analoga.
\paragraph{Val\\}
Ordförande och 0-4 övriga väljs av sektionsmötet.
\paragraph{Mandatperiod\\}
Mandatperioden är samma som verksamhetsåret. 
\subsubsection{Datas vargbröder}
\paragraph{Uppdrag\\}
Främja vargtischor, lone wolf t-shirts och vargen-månen-örnen-indianen t-shirts på sektionen 
\paragraph{Val\\}
Nya medlemmar väljs av föreningens medlemmar väljs av föreningens ordförande, och kanske dess medlemmar.
\paragraph{Mandatperiod\\}
Medlemskapet är på livstid. 
\newpage
%Valberedning%%%%%%%%%%%%%%%%%%%%%%%%%%%%%%%%%%%%%%%%%%%%%%%%%%%%%%%%%%
\section{Valberedning}
\subsection{Föreningsrepresentanter}
Nedan listade föreningar skall representeras av en föreningsmedlem ur respektive förening. Då sektionsförening eller SND blir valberedd har dessa rätt till 1-3 platser. Intresseförening som blir valberedd har rätt till 1 plats. 
\begin{itemize}
  \item DRUST 
  \item DAG 
  \item Delta 
  \item D6 
  \item DnollK 
  \item SND 
\end{itemize}
\subsection{Årskursrepresentanter}
\begin{itemize}
  \item 0-3 årskursrepresentanter utses av sammankallande. 
  \item Årskursrepresentanter skall ej inneha annan förtroendepost inom teknologsektionen. 
\end{itemize}
\subsection{Nomineringbarhet}
Valberedningsmedlem är ej nomineringsbar.
\subsection{Giltighet}
Valberedningen är giltig om ovanståaende uppfylls. Om ovanståaende ej uppfylls är det upp till sektionsmötet att besluta om valberedningens giltighet.
\end{document}
