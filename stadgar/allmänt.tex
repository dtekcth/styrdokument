\section{Allmänt}
\subsection{Ändamål}
\subsubsection{Föreningsform}
Datateknologsektionen vid Chalmers, härmed benämnd teknologsektionen, är en ideell förening bestående av studerande vid utbildningsprogrammet för Datateknik civilingenjör vid Chalmers.
Teknologsektionen är fackligt, partipolitiskt och religiöst oberoende.
\subsubsection{Uppdrag}
Teknologsektionen har som uppdrag att föra studiebevakning för sektionens medlemmar och verka för att varje medlem skall kunna tillgodogöra sin utbildning.
Teknologsektionen ska också verka för att varje medlem ska vara väl förberedd för arbetslivet, vara psykosocialt och fysiskt trygg samt känna en studiesocial gemenskap under sin tid som teknolog.
\subsection{Medlemmar}
Medlemmar defineras i \secref{sec:medlemmar}.
\subsection{Verksamhetsår}
Teknologsektionens verksamhetsår löper från och med den första maj. Ordinarie mandatperiod är 1:a maj - 30:e april.
\subsection{Räkenskapsår}
Teknologsektionens och dess kommittéers räkenskapsår löper från 1:a maj till 30:e april.
\subsection{Beslut}
\label{sec:beslut}
\subsubsection{Majoritetsdefinitioner}
Beslut kan antas med enkel, kvalificerad eller överväldigande majoritet.
Enkel majoritet betyder att 50\% av räknade röster krävs för att beslut ska bifallas.
kvalificerad majoritet betyder att $66.\bar{6}$\% av räknade röster krävs för att beslut ska bifallas.
Överväldigande majoritet betyder att 75\% av räknade röster krävs för att beslut ska bifallas.
Om inget annat är fastställt i denna stadga lyder enkel majoritet för beslutet i fråga.
\subsubsection{Rösträkning}
En giltig röst är en lagd röst antingen för eller emot ett beslut.
Vid räkning av lagda röster kan antingen avslappnad eller strikt rösträkning användas.
Avslappnad rösträkning innebär att enbart giltiga röster räknas.
Strikt rösträkning innebär att giltiga röster räknas och att icke giltiga röster räknas som en röst emot beslutet.
Om inget annat är fastställd i denna stadga lyder avslappnad rösträkning för beslutet i fråga.
%Skyddshelgon och sektionsfärger%%%%%%%%%%%%%%%%%%%%%%%%%%%%%%%%%%%%%%%
\subsection{Skyddshelgon}
Teknologsektionens skyddshelgon är Hacke Hackspett.
\subsection{Sektionsfärg}
Teknologsektionens färg är orange.
