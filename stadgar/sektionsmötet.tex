\section{Sektionsmötet}
\subsection{Sammanträden}
Det ska hållas fyra ordinarie sektionsmöten, ett per läsperiod, under verksamhetsåret. Utöver detta kan extra sektionsmöten hållas.
\subsection{Utlysande}
\subsubsection{Kallelse}
Sektionsmötet sammanträder på kallelse av talhenspresidiet eller av sektionsstyrelsen.
\subsubsection{Utlysningsrätt}
Rätt att hos sektionsstyrelsen eller talhenspresidiet begära utlysande av sektionsmöte tillkommer ledamot i sektionsstyrelsen, inspektor, kårens inspektor, kårstyrelsen, teknologsektionens lekmannarevisorer eller minst 25 av teknologsektionens medlemmar. Sådant möte ska hållas inom tio läsdagar.
\subsubsection{Utlysning}
\label{sec:sektionsmote_utlysande}
Sektionsmöte ska utlysas genom att kallelse anslås minst fem läsdagar i förväg.
Kallelse ska innehålla samtliga inkomna motioner, propositioner och bemötanden.
\subsection{Åligganden}
\subsubsection{Ordinarie mandatperiod}
Senast dagen före ordinarie mandatperiods början ska följande behandlas på sektionsmöte:
\begin{itemize}
\item Omfördelning av sektionens och dess organs tillgångar.
\item Val av sektionsstyrelse.
\item Val av lekmannarevisorer.
\item Val av inspektor om så är aktuellt.
\end{itemize}
\subsubsection{Före verksamhetsår}
Senast dagen före verksamhetsårets början ska följande behandlas på sektionsmöte:
\begin{itemize}
\item Sektionsavgift för de två kommande terminerna.
\item Fastställande av preliminär budget för nästkommande verksamhetsår.
\end{itemize}
\subsubsection{Sex månader efter verksamhetsårets början}
Senast sex månader efter verksamhetsårets början ska följande behandlas på sektionsmöte:
\begin{itemize}
\item Sektionens års- och revisionsberättelse för föregående verksamhetsår.
\item Beslut om ansvarsfrihet.
\item Fastställande av budget för innevarande verksamhetsår.
\end{itemize}
\subsection{Beslut}
\subsubsection{Beslutsmässighet}
Sektionsmötet är beslutsmässigt om mötet är behörigt utlyst enligt \secref{sec:sektionsmote_utlysande}.
\subsubsection{Kriterier}
Om färre än 40 medlemmar är närvarande då beslut ska fattas, kan detta endast ske om ingen yrkar på bordläggning. Detsamma gäller beslut i frågor som ej har varit anslagna fem läsdagar i förväg.
\subsection{Motion}
Medlem som önskar ta upp fråga på föredragningslistan ska anmäla detta skriftligen till sektionsstyrelsen och Talhenspresidiet senast sju läsdagar före sektionsmöte.
\subsubsection{Proposition}
En proposition är en motion som är inskickad av sektionsstyrelsen.
Propositioner följer inte kravet av att skickas in till sektionsstyrelsen eller Talhenspresediet minst sju läsdagar innan sektionsmöte.
\subsection{Överklagande}
Beslut av sektionsmötet som strider mot kårens eller sektionens stadga, reglemente, ekonomiska reglemente eller policy får undanröjas av kårfullmäktige. Sådant beslut ska tas upp till prövning om det begärs av en kårmedlem då det rör kårens stadga, eller sektionsmedlem då det rör teknologsektionens stadga, reglemente, ekonomiska reglemente eller policy.
\subsection{Omröstning}
\subsubsection{Fullmakt}
Röstning med fullmakt får ej ske.
\subsubsection{Omröstningsförfarande}
Omröstning ska ske öppet, om ej sluten votering begärs.
\subsubsection{Lika röstutfall}
Vid lika röstutfall äger mötesordförande utslagsröst, utom vid personval då lotten avgör.
\subsubsection{Flertal förslag}
Då flera förslag ställs mot varandra ska röstningsförfarandet fastslås innan omröstning påbörjas.
\subsubsection{Beslut}
Alla frågor som behandlas på sektionsmötet avgörs i enlighet med \secref{sec:beslut} i stadgan.
\subsection{Rättigheter}
\subsubsection{Närvarorätt}
Närvarorätt tillkommer medlem, hedersmedlem, stödmedlem, kårledningsledamöter, inspektor, kårens inspektor, lekmannarevisorer samt av mötet adjungerade icke-medlemmar.
\subsubsection{Yttranderätt}
Yttranderätt tillkommer medlem, hedersmedlem, kårledningsledamöter, inspektor, kårens inspektor, lekmannarevisorer samt av mötet adjungerade icke-medlemmar.
\subsubsection{Förslagsrätt}
Förslagsrätt tillkommer medlem, inspektor samt av mötet adjungerade icke-medlemmar.
\subsubsection{Rösträtt}
Rösträtt tillkommer medlem.
\subsubsection{Fråntagande av rättigheter}
Sektionsmötet kan temporärt frånta rättigheter från en person enbart i diskussion och beslut som berör den individen under personval och avsättning. 
\subsection{Protokoll}
Sektionsmötesprotokoll ska justeras av två av mötet valda justeringsmän. Justerat protokoll ska anslås senast tio läsdagar efter mötet.
