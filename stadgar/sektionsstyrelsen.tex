\section{Sektionsstyrelsen}
\subsection{Befogenheter}
Sektionsstyrelsen handhar i överensstämmelse med denna stadga, befintligt reglemente, befintligt ekonomiskt reglemente samt av sektionsmötet fattade beslut den verkställande ledningen av sektionens verksamhet.
\subsection{Sammansättning}
Sektionsstyrelsen består av:
\begin{itemize}
\item Ordförande
\item Vice ordförande
\item Kassör
\item Sekreterare
\item Studerande arbetsmiljöombud (SAMO)
\item i reglementet fastställda ledamöter
\end{itemize}
Sektionsstyrelsens ordförande och kassör ska båda vara myndiga.
\subsection{Rättigheter}
Sektionsstyrelsen äger rätt att i namn och emblem använda teknologsektionens namn och dess symboler i enlighet med Chalmers Studentkårs policies.
\subsection{Ansvarighet}
Sektionsstyrelsen ansvarar inför sektionsmötet för teknologsektionens verksamhet och ekonomi.
\subsection{Firmateckning}
Sektionsstyrelsens ordförande och sektionsstyrelsens kassör tecknar teknologsektionens firma i förening.
\subsection{Styrelsemöte}
Sektionsstyrelsen sammanträder minst tre gånger per läsperiod.
\subsection{Utlysande}
\subsubsection{Kallelse}
Sektionsstyrelsen sammanträder på kallelse av ordförande eller vice ordförande i sektionsstyrelsen.

Ledamot av sektionsstyrelsen äger rätt att hos vice ordförande i sektionsstyrelsen begära utlysande av styrelsemöte.
Sådant möte ska hållas inom 5 läsdagar.
\subsection{Beslutförhet}
Sektionsstyrelsen är beslutsmässigt när minst 50\% av ledamöter är närvarande. Ordförande eller vice ordförande ska närvara.
\subsection{Överklagande}
Beslut av sektionsstyrelsen som strider mot kårens eller teknologsektionens stadga, reglemente, ekonomiska reglemente samt policy får undanröjas av kårens fullmäktige. Sådant beslut ska tas upp till prövning om det begärs av en kårmedlem då det rör kårens stadga, eller av teknologsektionsmedlem då det rör sektionens stadga, reglemente, ekonomiska reglemente eller policy.
\subsection{Protokoll}
Protokoll ska föras vid styrelsemöte, justeras av två medlemmar av
sektionsstyrelsen och anslås senast fem läsdagar efter mötet.

