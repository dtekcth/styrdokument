\documentclass{dtek}



\title{Policy: Incidenthantering}

\makeheadfoot

\begin{document}
\maketitle

\section{Definitioner}
En incident kan sträcka sig från en liten händelse till en mycket allvarlig situation. Beroende på incidentens karaktär kommer således olika åtgärder tas. Nedan följer definitioner av några olika kategorier av incidenter som kan inträffa.

\subsection{Övertramp av styrdokument}
Kan vara om en sektionsmedlem eller kommitté bryter mot sektionens och kårens regler, stadgar, policyer eller dispositionsavtal. Exempel på detta är om en medlem inte gjort sitt nollanstäd, städat dåligt efter arrangemang, inte festanmält arrangemang eller liknande.

\subsection{Trakasserier och kränkande handlingar}
Trakasserier är ett agerande som kränker någons värdighet och/eller kan men måste inte kunna kopplas till någon av de sju diskrimineringsgrunderna. Det är alltid upp till personen som blivit utsatt att avgöra om en handling var trakasserande. Det kan vara allt från våldsamma handlingar, oönskade närmanden eller en taskig kommentar. 

\subsection{Gruppdynamikproblem}
Problematik inom eller mellan grupper, som förvärrar gruppens arbete eller medlemmarnas mående. Det kan bero på organisatoriska problem eller beteenden hos individer i gruppen.

\subsection{Skada av material eller person}
Allvarsnivån av dessa incidenter kan variera mycket. Exempel på incidenter som passar in här kan vara inbrott, stölder, vattenläckage eller personskada.

\subsection{Dödsfall}
Detta kan vara att sektionsstyrelsen hör ett rykte om att en kårmedlem har avlidit, att högskolan informerar styrelsen om det eller att ett dödsfall händer i anknytning till sektionens eller kårens verksamhet.

\subsection{Disciplinärende och straff}
Varken Datateknologsektionen eller Chalmers Studentkår hanterar
disciplinärenden. Disciplinärenden, såsom fusk på tentor, kränkningar, trakasserier och dylikt som allvarligt strider mot Chalmers policies eller som
grovt kan skada Chalmers rykte hanteras också av Chalmers tekniska högskola.
\section{Arbetsgång efter kännedom av incident}
Vid incidenter bör agerandet variera och anpassas till vilken typ- samt allvarlighetsgrad incidenten har. Nedan följer en del rekommendationer för agerande, observera att dessa endast är riktlinjer och att det kan uppkomma unika situationer som kräver annat agerande. 

\textbf{Rapportör} - Den som rapporterat en incident. \\
\textbf{Rapporterad} - Den som blivit rapporterad för en incident. 

\subsection{Gemensamt för varje fall}

\begin{itemize}
    \item Samla information kring händelsen och prata med eventuella inblandade, tex hör med vittnen eller ta hjälp av styrdokument/stadgar. 
    \item Boka in möte med de inblandade, börja alltid med rapportören och ha sen med övriga inblandade.
    \item Avgör och hör med rapportören samt andra inblandade om det är lämpligt att ha separata eller gemensamma möten. Ha alla parters bekvämlighet i åtanke. 
    \item Förbered ett underlag inför mötet med en tydlig struktur för vad ni vill få fram och få besvarat. 
    \item Anteckna noggrant samtliga parters återberättelser från deras perspektiv. Viktigt att fråga varje part om detta känns bekvämt och hålla det anonymt. 
    \item Förklara hur styrelsens process kommer fortgå efter mötet. 
    \item Ta hjälp av er kårlednings-kontakt eller kårens SO om det behövs (eventuellt vice kårordförande). 
    \item Ta hjälp av programledning om det behövs eller anses lämpligt.
    \item Meddela inblandade om det bestämda utfallet, både rapportören och den rapporterade. 
\end{itemize}

\subsection{Speciellt agerande vid olika typer av fall}

\subsubsection{Övertramp av styrdokument}
\begin{itemize}
    \item Diskutera tillsammans i sektionsstyrelsen och kontakta sektionens eller studentkårens inspektor om ni behöver hjälp att tolka stadgarna. Är det svårt att avgöra om ett övertramp har skett bör det alltid diskuteras med kårledningen. 
\end{itemize}

\subsubsection{Trakasserier och kränkande handlingar}
\begin{itemize}
    \item Om fler blivit utsatta avgör om gemensamma möten är lämpligt eller ej. 
    \item Tänk på att inte lova något före ni haft möte med alla inblandade och diskuterat inom styrelsen.
    \item Om den utsatte inte vill gå vidare med händelsen, avgör om ni bör agera oavsett. Gör en bedömning av händelsens allvarlighetsgrad. Om ni bedömer att det kan påverka någons säkerhet på sektionen kan ni välja att agera även om den utsatte inte önskar det.
    \item Vill ni gå vidare med händelsen, prata med den utsatte och förklara detta. Försök att hitta ett gemensamt sätt om hur hanteringen ska se ut. Till exempel kan ni höra om det känns bekvämt att gå vidare så länge den utsatte förblir anonym.
    \item Vid mindre allvarliga incidenter behöver nödvändigtvis inte hela styrelsen diskutera. Råder det tvivel om passande konsekvenser eller meningsskiljaktigheter bör ni kontakta kårens sociala enhetens ordförande för rådgivning.
    \item Avgör om en polisanmälan är lämplig i samråd med den utsatte. Kontakta kårens sociala enhetens ordförande för stöd. 
\end{itemize}

\subsubsection{Gruppdynamiksproblem}
\begin{itemize}
    \item Ha möte med alla som upplevt situationen problematisk. Fråga vad de tror är grunden för problematiken samt vad personerna tror kan hjälpa. Orsaken kan vara allt från ett bråk mellan två personer till två grupper som inte kommer överens. 

    \item Fastslå en handlingsplan om nödvändigt. Ta hänsyn till vad gruppmedlemmarna, särskilt ordförande,  tror hade hjälpt i situationen och ha deras feedback i åtanke i eventuell handlingsplan. Råder det tvivel om  vad styrelsen kan göra eller om styrelsen har delade åsikter bör ni kontakta sociala enhetens  ordförande för rådgivning. Tänk på att ge ordförande eller hela gruppen kontinuerlig hjälp med att följa handlingsplanen.
\end{itemize}

\subsubsection{Skada av material eller person}
\begin{itemize}
    \item Vid pågående händelse bör agerande ske skyndsamt: 
    \begin{itemize}
        \item Avgör om polis eller ambulans bör kontaktas. Det går även att kontakta  
        chalmersvakten på 031- 772 44 99.
        \item Om styrelsen inte själva är närvarande, försök få reda på vad som händer genom någon närvarande. 
    \end{itemize}
    \item Vid blåljus (t.ex. polis eller ambulans) som kommer till Chalmers bör kårordförande kontaktas direkt.
    \item Om allvarlig material- eller personskada sker ska styrelsen kontakta högskolans säkerhetssamordnare samt informera sociala enhetens ordförande.
    \item Har händelsen redan inträffat eller om krishanteringsstadiet är över, ta reda på vilka som är inblandade och eventuella vittnen.
    \item Eftersom graden av allvar och möjliga händelser varierar mycket inom denna kategori finns  inte endast en version av steg att ta efter den initiala hanteringen. Därför bör sociala enhetens ordförande kontaktas om det råder tvivel kring hur situationen ska hanteras.
\end{itemize}

\subsubsection{Dödsfall}
\begin{itemize}
    \item När sektionsstyrelsen får vetskap om ett förmodat dödsfall av en sektionsmedlem ska sociala  enhetens ordförande kontaktas. Sociala enhetens ordförande kommer förmedla  informationen till högskolan men styrelsen kan även kontakta programansvarig för att sprida  informationen till denne snabbare.
    \item En eller flera representanter från styrelsen förväntas närvara om högskolan tillsammans med kåren arrangerar en minnesstund. Sociala enhetens ordförande kontaktar SAMO gällande detta.
    \item Sker ett förmodat dödsfall i anknytning till sektionens eller studentkårens verksamhet ska ambulans tillkallas och kårordförande samt sociala enhetens ordförande informeras direkt. Styrelsen har inget vidare ansvar i att hantera situationen ytterligare om inte instruktioner ges från  kårordförande eller sociala enhetens ordförande.
    \item Finns det medlemmar som är i behov av stöd bör styrelsen ge dem stöd i stunden till den nivå  som känns bekvämt för styrelsen. Detta kan innefatta att sitta med personer som är  påverkade av situationen tills vidare hjälp kan ges till dessa eller att trösta sina medmänniskor till den grad man är bekväm med.
    \item Sociala enhetens ordförande, studentpräst eller kurator kan alltid kontaktas för att koordinera akut stöd, styrelsen ska inte ta på sig till exempel stödsamtal eller annan typ av vård av chock.
    \item Om akut stöd ej behövs kan styrelsen hänvisa personer som blivit påverkade till att ta kontakt i efterhand med sociala enhetens ordförande eller någon av stödfunktionerna på högskolan, förslagsvis studentprästen eller kurator.  
\end{itemize}
\section{Påföljder}
Nedan listas vilken typ av åtgärder styrelsen kan använda efter det att en incident har skett, utan inbördes ordning. Dessa måste tidsbestämmas, undantag gäller ``Rött kort'' där en städuppgift som tilldelas av DRust ska utföras för att bli av med straffet. Ett straff får inte tidsbestämmas till över ett år.

\subsection{Definitioner av följder}
\begin{itemize}
    \item \textbf{Varning}\\
    En varning där personen tydligt blir tillsagd vad personen gjort fel. Varningen är tidsbegränsad så länge det anses vara passande då varningen utfärdas och sparas internt under tiden. Detta för att använda som underlag i de fall personen är involverad i ytterligare incidenter. Personen ska informeras om hur länge all information kopplad till personen sparas. 

    \item \textbf{Representationsförbud}\\
    Innebär att den rapporterade inte får lov att representera sektionen i något sammanhang. Personen får inte lov att t.ex. bära kläder kopplade till sektionen eller vara funktionär i sådana miljöer där personen syns. Att vara funktionär i bakgrunden, t.ex. laga mat för ett arrangemang där person i fråga inte syns utåt är fortfarande accepterat.
    \item \textbf{Förlorad åtkomst till teknologsektionens förråd}\\
    Att förlora åtkomsten till förråd kan endast ges till de i kommittéer med tillgång till förråd och innebär att personen inte längre får vistas där. Personen i fråga kan även bli av med sin nyckel eller access om styrelsen anser det lämpligt.
    \item \textbf{Förlorad åtkomst till teknologsektionens lokaler}\\
    Personen får inte längre vistas i någon av Datatekonologsektionens lokaler. Detta medför indragen access till Basen. Dock får personen i fråga fortfarande delta i teknologsektionens arrangemang även om de sker i teknologsektionens lokaler. 
    \item \textbf{Avstängning från sektionens arrangemang}\\
    Innebär att vederbörande inte får delta under något av teknologsektionens arrangemang.
    \item \textbf{``Rött kort''}\\
    En mildare version av "förlorad åtkomst till teknologsektionens lokaler" som börjar verka först två läsveckor efter styrelsens beslut. Detta straffet avslutas efter att den berörda utfört en städuppgift tilldelad av DRust.
%    \item \textbf{Avsättning}\\
%    Att bli avsatt innebär att personen i fråga förlorar sin post i relevanta kommitteer och denna post vakantsätts. Personen ska lämna tillbaka ev. nycklar, bli fråntagen åtkomst till ev. förråd och får inte längre representera teknologsektionen.  
\end{itemize}

\subsection{Riktlinjer vid policyöverträdelser}
\begin{itemize}
    \item\textbf{Varning}\\
    Kan användas vid incidenten är av mildare grad eller i övrigt anses vara nog. Varning är endast lämpligt då den rapporterade inte blivit varnad tidigare. 
    \item \textbf{Representationsförbud}\\
    Kan användas i situationer där den rapporterade bär representationsplagg vid överträdelsen. Representationsförbud kan också vara aktuellt då personen anses representera sektionen på ett negativt sätt. 
    \item \textbf{Förlorad åtkomst till teknologsektionens förråd}\\
    Kan användas i situationer där överträdelsen sker i samband med att personen i fråga haft ansvar för lokalen och/eller befunnit sig i lokalen. 
    Kan användas då incidenten på något sätt kan kopplas till sektionens förråd. T.ex. att personen befann sig där vid incidenten eller missbrukat sitt ansvar över förrådet. 
    \item \textbf{Förlorad åtkomst till teknologsektionens lokaler}\\
    Kan tillämpas då personen skadat och/eller smutsat ner någon av sektionens lokaler. Det kan även användas om personen stått som ansvarig under t.ex. ett arrangemang. Kan också användas i kombination med avstängning från arrangemang för förstärkt straff. 
    \item \textbf{Avstängning från teknologektionens arrangemang}\\
    Kan vara lämpligt då incidenten skett under ett av sektionens arrangemang. T.ex. någon form av kränkning eller annat som skapat obehag hos andra deltagare. 
    \item \textbf{``Rött kort''}\\
    Bör tilldelas Nollan vid missat nollanstäd men undantag att Nollan fyllt i en lämplig ursäkt för att missa städet. Är också lämpligt om andra sektionsmedlemmar inte städat tillräckligt efter arrangemang i sektionens lokaler. 
%    \item \textbf{Avsättning}\\
%    Kan inte ges av styrelsen utan ska tas till sektionsmötet. För att ta ärendet till sektionsmötet krävs två tredjedelars majoritet enligt sektionens stadga. I första hand ska styrelsen be personen att avgå och på så sätt behöver ärendet inte tas till sektionsmötet. Praxis bör vara att stänga av från lokaler och arrangemang under tiden till nästa sektionsmöte. Styrelsen bör rådfråga kårledningen i ärenden som gäller avsättning.
\end{itemize}

\section{Dokumentation}
\subsection{Protokollföring och sekretess}
\begin{itemize}
  \item Diskussionsprotokoll och fullständiga protokoll skall endast sparas i syfte att möjliggöra för framtida incidenter att bedömas rättvist och i enlighet med tidigare åtgärder.
  \item D-styret har rätt att vägra offentliggöra fullständiga protokoll, om inte särskilda skäl föreligger. Dock inte beslutsprotokoll, som alltid anslås anonymiserade.
  \item D-styret har rätt att makulera personuppgifter och kännetecknande detaljer i protokoll, efter att åtgärdens giltighetstid löpt ut. Dock tidigast efter det att ärendet ej längre kan överklagas.
  \end{itemize}
\subsection{Rekommendationer för protokollförande}
\begin{itemize}
  \item Undvik att skriva personuppgifter och kännetecknande detaljer i mail eller annan plats som utomstående av misstag lätt kan ta del av.
  \item D-styret bör föra en intern lista över personer som berörs av åtgärder som inte lagts till handlingarna. Information från denna lista bör enbart delges de som behöver den för att åtgärden skall kunna verkställas.
\end{itemize}

\section{Överklagan}
Överklagan sker enligt formulär för överklagan av styrelsebeslut.
\subsection{Omprövning}
  Begäran om omprövning görs till D-styret, SAMO eller anonymt till någon av medlemmarna i D-styret. Om begäran görs anonymt till en medlem i D-styret skall denna rapportera begäran om omprövan till D-styret, men anförtros då att inte delge vem som begärt omprövningen.
%\section{Exempel}
%TODO
%vore bra med ett exempel som vi kan luta oss tillbaka på när vi tar beslut angående ett riktigt straff vi ger ut för att uppskatta hur mycket mildare/värre det någon gjort är.

\end{document}