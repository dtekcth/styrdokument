\section{Arbetsgång efter kännedom av incident}
Vid incidenter bör agerandet variera och anpassas till vilken typ- samt allvarlighetsgrad incidenten har. Nedan följer en del rekommendationer för agerande, observera att dessa endast är riktlinjer och att det kan uppkomma unika situationer som kräver annat agerande. 

\textbf{Rapportör} - Den som rapporterat en incident. \\
\textbf{Rapporterad} - Den som blivit rapporterad för en incident. 

\subsection{Gemensamt för varje fall}

\begin{itemize}
    \item Samla information kring händelsen och prata med eventuella inblandade, tex hör med vittnen eller ta hjälp av styrdokument/stadgar. 
    \item Boka in möte med de inblandade, börja alltid med rapportören och ha sen med övriga inblandade.
    \item Avgör och hör med rapportören samt andra inblandade om det är lämpligt att ha separata eller gemensamma möten. Ha alla parters bekvämlighet i åtanke. 
    \item Förbered ett underlag inför mötet med en tydlig struktur för vad ni vill få fram och få besvarat. 
    \item Anteckna noggrant samtliga parters återberättelser från deras perspektiv. Viktigt att fråga varje part om detta känns bekvämt och hålla det anonymt. 
    \item Förklara hur styrelsens process kommer fortgå efter mötet. 
    \item Ta hjälp av er kårlednings-kontakt eller kårens SO om det behövs (eventuellt vice kårordförande). 
    \item Ta hjälp av programledning om det behövs eller anses lämpligt.
    \item Meddela inblandade om det bestämda utfallet, både rapportören och den rapporterade. 
\end{itemize}

\subsection{Speciellt agerande vid olika typer av fall}

\subsubsection{Övertramp av styrdokument}
\begin{itemize}
    \item Diskutera tillsammans i sektionsstyrelsen och kontakta sektionens eller studentkårens inspektor om ni behöver hjälp att tolka stadgarna. Är det svårt att avgöra om ett övertramp har skett bör det alltid diskuteras med kårledningen. 
\end{itemize}

\subsubsection{Trakasserier och kränkande handlingar}
\begin{itemize}
    \item Om fler blivit utsatta avgör om gemensamma möten är lämpligt eller ej. 
    \item Tänk på att inte lova något före ni haft möte med alla inblandade och diskuterat inom styrelsen.
    \item Om den utsatte inte vill gå vidare med händelsen, avgör om ni bör agera oavsett. Gör en bedömning av händelsens allvarlighetsgrad. Om ni bedömer att det kan påverka någons säkerhet på sektionen kan ni välja att agera även om den utsatte inte önskar det.
    \item Vill ni gå vidare med händelsen, prata med den utsatte och förklara detta. Försök att hitta ett gemensamt sätt om hur hanteringen ska se ut. Till exempel kan ni höra om det känns bekvämt att gå vidare så länge den utsatte förblir anonym.
    \item Vid mindre allvarliga incidenter behöver nödvändigtvis inte hela styrelsen diskutera. Råder det tvivel om passande konsekvenser eller meningsskiljaktigheter bör ni kontakta kårens sociala enhetens ordförande för rådgivning.
    \item Avgör om en polisanmälan är lämplig i samråd med den utsatte. Kontakta kårens sociala enhetens ordförande för stöd. 
\end{itemize}

\subsubsection{Gruppdynamiksproblem}
\begin{itemize}
    \item Ha möte med alla som upplevt situationen problematisk. Fråga vad de tror är grunden för problematiken samt vad personerna tror kan hjälpa. Orsaken kan vara allt från ett bråk mellan två personer till två grupper som inte kommer överens. 

    \item Fastslå en handlingsplan om nödvändigt. Ta hänsyn till vad gruppmedlemmarna, särskilt ordförande,  tror hade hjälpt i situationen och ha deras feedback i åtanke i eventuell handlingsplan. Råder det tvivel om  vad styrelsen kan göra eller om styrelsen har delade åsikter bör ni kontakta sociala enhetens  ordförande för rådgivning. Tänk på att ge ordförande eller hela gruppen kontinuerlig hjälp med att följa handlingsplanen.
\end{itemize}

\subsubsection{Skada av material eller person}
\begin{itemize}
    \item Vid pågående händelse bör agerande ske skyndsamt: 
    \begin{itemize}
        \item Avgör om polis eller ambulans bör kontaktas. Det går även att kontakta  
        chalmersvakten på 031- 772 44 99.
        \item Om styrelsen inte själva är närvarande, försök få reda på vad som händer genom någon närvarande. 
    \end{itemize}
    \item Vid blåljus (t.ex. polis eller ambulans) som kommer till Chalmers bör kårordförande kontaktas direkt.
    \item Om allvarlig material- eller personskada sker ska styrelsen kontakta högskolans säkerhetssamordnare samt informera sociala enhetens ordförande.
    \item Har händelsen redan inträffat eller om krishanteringsstadiet är över, ta reda på vilka som är inblandade och eventuella vittnen.
    \item Eftersom graden av allvar och möjliga händelser varierar mycket inom denna kategori finns  inte endast en version av steg att ta efter den initiala hanteringen. Därför bör sociala enhetens ordförande kontaktas om det råder tvivel kring hur situationen ska hanteras.
\end{itemize}

\subsubsection{Dödsfall}
\begin{itemize}
    \item När sektionsstyrelsen får vetskap om ett förmodat dödsfall av en sektionsmedlem ska sociala  enhetens ordförande kontaktas. Sociala enhetens ordförande kommer förmedla  informationen till högskolan men styrelsen kan även kontakta programansvarig för att sprida  informationen till denne snabbare.
    \item En eller flera representanter från styrelsen förväntas närvara om högskolan tillsammans med kåren arrangerar en minnesstund. Sociala enhetens ordförande kontaktar SAMO gällande detta.
    \item Sker ett förmodat dödsfall i anknytning till sektionens eller studentkårens verksamhet ska ambulans tillkallas och kårordförande samt sociala enhetens ordförande informeras direkt. Styrelsen har inget vidare ansvar i att hantera situationen ytterligare om inte instruktioner ges från  kårordförande eller sociala enhetens ordförande.
    \item Finns det medlemmar som är i behov av stöd bör styrelsen ge dem stöd i stunden till den nivå  som känns bekvämt för styrelsen. Detta kan innefatta att sitta med personer som är  påverkade av situationen tills vidare hjälp kan ges till dessa eller att trösta sina medmänniskor till den grad man är bekväm med.
    \item Sociala enhetens ordförande, studentpräst eller kurator kan alltid kontaktas för att koordinera akut stöd, styrelsen ska inte ta på sig till exempel stödsamtal eller annan typ av vård av chock.
    \item Om akut stöd ej behövs kan styrelsen hänvisa personer som blivit påverkade till att ta kontakt i efterhand med sociala enhetens ordförande eller någon av stödfunktionerna på högskolan, förslagsvis studentprästen eller kurator.  
\end{itemize}