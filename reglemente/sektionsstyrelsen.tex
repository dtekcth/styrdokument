\section{Sektionsstyrelsen}

\subsection{Ansvarsområden}

\subsubsection{Det åligger sektionsstyrelsen att:}

\begin{itemize}
  \item verka för sammanhållningen mellan sektionsmedlemmarna och verka för deras gemensamma intressen 
  \item leda sektionens arbete 
  \item övervaka genomförandet av sektionsmötesbeslut och se till att de verkställs 
  \item framlägga budget med förslag på sektionsavgift till sektionsmötet 
  \item framlägga preliminär verksamhetsplan vid sista ordinarie vårmötet 
  \item lämna förslag på representanter till sektionens valberedning 
  \item utse representant från styrelsen till DatE-ITs styrelse
  \item lämna ett skriftligt bemötande till motioner som inkommit i tid senast 5 läsdagar innan sektionsmöte
\end{itemize}

\subsubsection{Det åligger sektionsordförande att:} 

\begin{itemize}
  \item teckna sektionens firma.
  \item se till att sektionens beslut verkställs.
  \item se till att ordförande i varje kommitté har tillgång till och kunskap om sektionens stadgar, reglemente och ekonomiska reglemente.
  \item se till att ordförande i relevanta kommittéer skriver en verksamhetsrapport inför varje brytpunkt.
  \item leda och övervaka arbetet inom sektionsstyrelsen.
  \item till varje sektionsmöte kunna redogöra om sektionens verksamhet.
  \item föra sektionens talan då något annat ej stadgats eller beslutats. 
  \item representera sektionen på kårledningsutskottet.
\end{itemize}

\subsubsection{Det åligger sektionens vice ordförande att:}

\begin{itemize}
  \item biträda ordföranden i dennes värv 
  \item i ordförandens frånvaro överta dennes åligganden 
  \item kontrollera så att arbetet sker i enighet med sektionens bestämmelser 
  \item representera sektionen på kårens nöjeslivsutskott
\end{itemize}

\subsubsection{Det åligger sektionskassör att:}

\begin{itemize}
  \item teckna sektionens firma
  \item se till att kassör i varje kommitté har tillgång till och kunskap om sektionens stadgar, reglemente och ekonomiska reglemente.
  \item se till att kassör i varje kommitté förstår och kan använda sektionens bokförings och redovisningssystem.
  \item i samråd med styrelsen upprätta budgetförslag till första ordinarie höstmötet.
  \item fortlöpande kontrollera sektionens räkenskaper och bokföring.
  \item till varje sektionsmöte kunna redogöra för sektionens ekonomiska ställning.
  \item representera sektionen på kårens sektionsekonomiforum.
\end{itemize}

\subsubsection{Det åligger vice sektionskassör att:}

\begin{itemize}
  \item biträda styrelsens kassör i dennes värv
  \item i styrelsens kassörs frånvaro överta dennes åligganden. 
        Dock får vice sektionskassör inte teckna firman.
\end{itemize}

\subsubsection{Det åligger sektionens sekreterare att:}

\begin{itemize}
  \item föra protokoll vid styrelsens möten och tillse att protokoll från såväl styrelse- som sektionsmöten anslås
  \item se till att material som inkommer till sektionen anslås eller på annat sätt förmedlas till berörda parter
  \item se till att sektionens styrdokument hålls uppdaterade i enlighet med sektionsmötes- och styrelsebeslut
  \item se till att samtliga dokument och andra resurser som sektionen tillhandahåller hanteras i enlighet med gällande lagstiftning.
  \item administera sektionens tekniska lösningar gällande kommunikation
\end{itemize}

\subsubsection{Det åligger sektionens SAMO att:}

\begin{itemize}
    \item verka för studenternas trivsel på sektionen
    \item bistå studenter i frågor kring fysisk och psykisk hälsa
    \item föra sektionens talan i frågor kring psykosocial och fysisk studie- och arbetsmiljö
    \item representera sektionen på kårens sociala utskott
\end{itemize}

\subsubsection{Det åligger styrelsens övriga ledamöter att:}

\begin{itemize}
  \item bistå styrelsen med information 
  \item aktivt deltaga i beslutsprocessen 
  \item redogöra för sin egen eller sin kommittés löpande verksamhet vid styrelsens möten 
\end{itemize}

\subsection{Insyn} 

Sektionsstyrelsen har full insyn i teknologsektionen alla organ och äger rätt att deltaga i deras möten med yttranderätt. 

\subsection{Ledamöter}
Styrelsens ledamöter är, utöver de i stadgarna uppräknade, även följande förtroendeposter. 
\begin{itemize}
  \item DRUST ordförande 
  \item DAG ordförande 
  \item Delta ordförande 
  \item D6 ordförande 
  \item DNollK ordförande
  \item DNS ordförande
  \item Vice sektionskassör
\end{itemize}

Samtliga ledamöter i styrelsen har närvaro-, yttrande-, röst- och förslagsrätt.

\subsubsection{Suppleant}
Då ledamot från DRUST, DAG, Delta, D6, DNollK eller DNS ej kan närvara vid möte har denne rätt att utse en suppleant ur samma kommitté.
Suppleant har närvaro-, yttrande-, röst- och förslagsrätt.

\subsection{Val}
Val till sektionsstyrelsen sker genom inval av presidiet och ordförande i DRUST, DAG, Delta, D6, DNollK och DNS.

\subsection{Omröstning} 

\begin{itemize}
  \item Röstning med fullmakt får ej ske. 
  \item Omröstning ska ske öppet.
  \item Vid lika utfall äger mötesordförande utslagsröst
  \item Då flera förslag ställs mot varandra ska röstningsförfarandet fastslås innan omröstning påbörjas
\end{itemize}

\subsection{Lekmannarevisorer}
Teknologsektionens lekmannarevisorer har rätt att medverka med närvaro-
och yttranderätt på styrelsemöten.
\newpage

