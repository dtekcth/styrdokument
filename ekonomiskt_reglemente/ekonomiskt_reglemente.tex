%LaTeX-inställningar%%%%%%%%%%%%%%%%%%%%%%%%%%%%%%%%%%%%%%%%%%%%%%%%%%%
% Kompliera med pdflatex!!!
\documentclass[a4paper, 10pt]{article}

\usepackage{hyperref}
\usepackage[utf8]{inputenc}
\usepackage[T1]{fontenc}
\usepackage[swedish]{babel}
\usepackage{graphicx}
%\usepackage{ae} %För riktiga fonts?
\usepackage{fancyhdr}
\usepackage{lastpage}
\topmargin -20.0pt
\headheight 56.0pt
\setcounter{secnumdepth}{5}

%FYLL I VID ÄNDRINGAR%%%%%%%%%%%%%%%%%%%%%%%%%%%%%%%%%%%%%%%%%%%%%%%%%%
\newcommand{\updated}{2021-08-29} %Insert




%Dokumentstart%%%%%%%%%%%%%%%%%%%%%%%%%%%%%%%%%%%%%%%%%%%%%%%%%%%%%%%%%
\begin{document}
\pagestyle{fancy}

%Header%%%%%%%%%%%%%%%%%%%%%%%%%%%%%%%%%%%%%%%%%%%%%%%%%%%%%%%%%%%%%%%%
\renewcommand{\headrule}{\vbox to 0pt{\hfill\hbox to 292pt{\hrulefill}}}
\lhead{
\raisebox{-25pt}[0pt][10pt]{\includegraphics[width=60pt]{dteklogo.pdf}}
\parbox[b]{200pt}{
\textbf{Datateknologsektionen}\\
Chalmers studentkår\\
Ekonomiskt reglemente}}
\rhead{ \flushright
Sidan \thepage\ av \pageref{LastPage}\\
Uppdaterad \updated}


%Footer%%%%%%%%%%%%%%%%%%%%%%%%%%%%%%%%%%%%%%%%%%%%%%%%%%%%%%%%%%%%%%%%
\renewcommand{\footrulewidth}{\headrulewidth}
\lfoot{\flushleft Datateknologsektionen\\
    Rännvägen 8\\
    412 58 Göteborg}
\cfoot{}
\rfoot{ \flushright styret@dtek.se\\
www.dtek.se}
\newpage

%Titlepage%%%%%%%%%%%%%%%%%%%%%%%%%%%%%%%%%%%%%%%%%%%%%%%%%%%%%%%%%%%%%
\vspace*{\fill}
\begin{center}
{\Huge \textbf{Ekonomiskt reglemente för Datateknologsektionen}}\\
\includegraphics[width=300pt]{dteklogo.pdf}
\\{\LARGE Chalmers, Göteborg}
\end{center}
\vspace*{\fill}
\begin{center}
{\LARGE Uppdaterad: \updated}
\end{center}
\vspace*{\fill}


\newpage
\setcounter{tocdepth}{1}
\tableofcontents
\newpage

%Allmänt%%%%%%%%%%%%%%%%%%%%%%%%%%%%%%%%%%%%%%%%%%%%%%%%%%%%%%%%%%%%%%%
\section{Allmänt}
\subsection{}
Detta reglemente har skapats för att skapa tydlighet över vad sektionens pengar får och inte får användas till. Det är därmed tänkt att vara ett hjälpmedel för styrelsen, revisorerna och sektionens kommittéer i deras arbete. Grundfilosofin är att sektionens pengar är till för sektionens samtliga medlemmar, vilket i praktiken innebär att arrangemang som inte är öppna för dessa inte får belasta sektionens ekonomi. Viktigt att tänka på vid tolkandet utav reglementet är att det inte har utformats för att i onödan begränsa kommittéers möjligheter eller lust till roliga arrangemang och upptåg.
\subsection{Prisbasbelopp}
\begin{itemize}
  \item Det prisbasbelopp som skall användas under hela verksamhetsåret är det prisbasbelopp som är aktuellt vid verksamhetsårets början.
  \item Delar av prisbasbelopp skall avrundas till närmaste hundratal. 
\end{itemize}

%D-Styret%%%%%%%%%%%%%%%%%%%%%%%%%%%%%%%%%%%%%%%%%%%%%%%%%%%%%%%%%%%%%%
\section{Styrelsen}
\subsection{}
Styrelsen är ansvariga för sektionens hela ekonomi
\subsection{}
Ordförande är skyldig att
\begin{itemize}
  \item Teckna sektionens firma
  \item Till varje sektionsmöte kunna redogöra om sektionens verksamhet
  \item Se till att ordförande i varje kommitté har tillgång till och kunskap om sektionens stadgar, reglemente och ekonomiska reglemente
  \item Se till att ordförande i komittéer skriver en verksamhetsrapport inför varje brytpunkt.
\end{itemize}
\subsection{}
Kassören är skyldig att
\begin{itemize}
  \item Teckna sektionens firma
  \item Se till att kassör i varje kommitté har tillgång till och kunskap om sektionens stadgar, reglemente och ekonomiska reglemente 
  \item Fortlöpande kontrollera sektionens räkenskaper och bokföring
  \item I samråd med styrelsen upprätta budgetförslag till första ordinarie höstmötet
  \item Till varje sektionsmöte kunna redogöra för sektionens ekonomiska ställning
  \item Utbilda nya sektionsfunktionärer i hur sektionens bokförings och redovisningssystem skall användas
\end{itemize}
\subsection{}
Styret har rättighet att inom ramen för styrets budget representera sektionen och ordna förtäring till styrets ledamöter i samband med styrelsemöten.

%Sektionskommittéer%%%%%%%%%%%%%%%%%%%%%%%%%%%%%%%%%%%%%%%%%%%%%%%%%%%%
\section{Kommittéer}
\label{sec:kommitteer}
\subsection{}
Ordförande i kommitté är skyldig att
\begin{itemize}
\item Kontinuerligt meddela kommitténs ekonomiska status till styret
\item Tillsammans med kommitténs kassör ansvara för att kommittén förvaltar sina tillgångar i enlighet med sektionens stadgar, reglementen och beslut.
\item Tillsammans med kommitténs kassör skriva ett bindande avtal med styret angående skuldfrågan vid felaktig bokföring
\end{itemize}
\subsection{}
\label{sec:kommittee_kassor}
Kassör i sektionskommitté är skyldig att
\begin{itemize}
\item Föra kassabok av sådan typ som godkänts av sektionens revisorer
\item Tillsammans med kommitténs ordförande ansvara för att kommittén förvaltar sina tillgångar i enlighet med sektionens stadgar, reglementen och beslut.
\item Tillsammans med kommitténs ordförande skriva ett förbindande avtal med Styret angående skuldfrågan vid felaktigt förd bokföring
\item På varje sektionsmöte redovisa kommitténs ekonomiska situation
\item Arkivera kommitténs bokföring, på en plats anvisad av Styret, så lång tid som föreskrivs för den organisationsform som datateknologsektionen är.
\item Lägga budget enligt \S\ref{sec:budget}
\end{itemize}
\subsection{}
För kommittéer som saknar kassör ansvarar istället styrelsens kassör för samtliga av punkterna i \S\ref{sec:kommittee_kassor} med undantag för att lägga budget vilket görs i samråd med kommittéens ordförande
\subsection{}
Varje kommitté skall påbörja en ny kassabok i starten av sitt verksamhetsår
\subsection{Budget}
\label{sec:budget}
Varje kommitté ska i början av året lägga en budget för hur dess pengar ska spenderas. Det är upp till styrelsens kassör att delge en mall för budget för att underlätta kommittéernas arbete. I budgeten ska det tydligt framgå hur pengarna ska spenderas (uppdelat på bokföringskonton) med en uppskattning för vilken månad utgifter och inkomster hamnar i. 

\subsubsection{}
Varje kommittés budgetförslag ska godkännas av styrelsens kassör. Som rapporterar vidare relevant information till sektionsmötet.


\subsection{Kapital}
\label{sec:sektionsforeningar_startkapital}
\subsubsection{Startkapital}
Kommittéer skall vid mandatperiodens början ha 0 prisbasbelopp i tillgångar. 
\subsubsection{Startlån}
Följande Kommittéer har rätt att låna 0,25 prisbasbelopp från sektionsstyrelsen i form av startlån: 
\begin{itemize}
  \item DRust
  \item DAG
  \item Delta
  \item D6
  \item DNollK
  \item DBus
  \item DKock
\end{itemize}
DAG har rätt låna ytterligare 1 prisbasbelopp i form av utökat startlån.
\subsubsection{Slutkapital}
Samtliga kommitteers tillgångar och skulder tillfaller sektionsstyrelsen vid mandatperiodens slut.
\subsubsection{Stödkapital}
Styret har rätt att utföra kortfristiga lån till sektionskommittéer för att betala kristiska skulder i väntan på ankommande kapital.
\subsection{Budget}
För arrangemang med förväntad omsättning överstigande 0.75 prisbasbelopp skall en noggrann budget upprättas och godkännas av styret innan arrangemanget får genomföras.

Dessa utgifter får ej överskrida kommitténs ekonomiska kapacitet. Den fria kostens tillagning får inte medräknas i arrangemangets tidsåtgång

%Äskning%%%%%%%%%%%%%%
\section{Äskning}
Medlem eller kommittéer som önskar extra ekonomiska medel till sin verksamhet eller arrangemang skall inkomma med önskemål och skäl till styret. Styret kan bevilja extra medel om beloppet understiger 0,25 prisbasbelopp, om beloppet är större måste först sektionsmöte tillfrågas.

%Förmåner%%%%%%%%%%%%%%
\section{Förmåner}

\subsection{Internrepresentation}
\label{sec:internreps}
Varje sektionskommitté och DNS har rättighet att varje verksamhetsår bekosta följande på sind ekonomi:
\begin{itemize}
    \item[-] En eller flera teambuildingaktiviteter för upp till 0,015 prisbasbelopp per person sammanlagt.
    \item[-] Överlämning för upp till 0,0020 prisbasbelopp per person som deltar.
    \item[-] Aspning och aspplagg för upp till 0,065 prisbasbelopp sammanlagt.
\end{itemize}

\subsection{Arbetskläder}

Kommitté får lägga upp till 0.025 prisbasbelopp per person på arbetskläder. Rimliga arbetskläder kan anses vara en overall eller motsvarande, t-shirt samt annan
tröja/piké.

\subsection{Fri kost}
Vid arrangemang för sektionen kan arbetsmat beviljas. Vilket belastar sektionens ekonomi med utgifter för inköp av fri kost, enligt följande tidsåtgång för arrangemanget, avrundat till närmaste femkrona:
\begin{itemize}
    \item 1–4 timmar: 0,00075 basbelopp per person.
    \item 4–8 timmar: 0,0015 basbelopp per person.
    \item Mer än 8 timmar: 0,00225 basbelopp per person.
\end{itemize}

Belstningen på ekonomin sker på 
Belastningen av ekonomin sker på kommittee Vid arrangemang kopplat till en specifik kommitté belastas kommittéens ekonomi

Ovanstående får inte överskrida kommitténs ekonomiska resurser. \\
Alkoholhaltiga drycker får ej bekostas med kommitténs pengar vid ovanstående arrangemang undantag gäller för teambuildingaktiviteter där alkoholhaltiga drycker får bekostastill upp till en tredjedel av teambuildingbudgeten.


%Sponsring%%%%%%%%%%%%%%%%%%%%%%%%%%%%%%%%%%%%%%%%%%%%%%%%%%%%%%%%%%%%%
\section{Sponsring}
\subsection{}
Kommittéer får inte söka sponsring utan samråd med datas
arbetsmarknadsgrupp, DAG.

%Fonder%%%%%%%%%%%%%%%%%%%%%%%%%%%%%%%%%%%%%%%%%%%%%%%%%%%%%%%%%%%%%%%%
\section{Fonder}
De fonder som beskrivs i denna del syftar till sparande av medel för framtida användande enligt respektive fonds syfte, antingen planerat eller spontant. Ett uttag ur en fond skall inte belasta Sektionsstyrelsens godkända budget, däremot skall inköp och investeringar vara föremål för redovisning i balans– respektive resultaträkning. Detta gäller om inget annat specificeras i fonden.
\subsection{Kapitalfonden}
\subsubsection{Syfte}
\label{sec:kapitalfond_syfte}
Syftet med kapitalfonden är att avlasta sektionens respektive de olika sektionskommittéernas ekonomi från stora investeringar. Pengarna skall användas till saker som har ett bestående värde, samt är till gagn för sektionens medlemmar direkt eller indirekt. Pengar skall ej användas till driftbidrag eller stöd för förgänglig verksamhet. Fonden skall ej användas till verksamhet som lokalfond är ämnad för.
\subsubsection{Förvaltning}
Fonden förvaltas av sektionens styrelse.
\subsubsection{Avsättning}
\begin{itemize}
\item En av styrelsen budgeterad summa som godkänts av sektionsmötet
\item All avkastning ifrån kapitalfonden tillförs kapitalfonden.
\end{itemize}
\subsubsection{Uttag}
\begin{itemize}
\item Sektionsstyrelsen har rätt att bevilja uttag ur fonden på belopp upp till totalt 0,25 prisbasbelopp per tillfälle. Uttag av belopp överstigande 0,25 prisbasbelopp skall godkännas av sektionsmöte innan medel utbetalas.
\item Sektionsstyrelsen äger inte rätt att ta ut mer än 50\% av fondens totala värde per tillfälle.
\item Sektionsmötet äger rätt att vid välmotiverat behov göra uttag över 0.25 prisbasbelopp eller i strid med syftet i \S\ref{sec:kapitalfond_syfte}
\end{itemize}

\subsection{Lokalfonden}
\subsubsection{Syfte}
\label{sec:lokalfond_syfte}
Syftet med lokalfonden är att säkra medel för underhåll och reparationer av sektionens lokaler, samt för inköp av inventarier. Pengarna skall användas till större reparationer och ommålningar av sektionslokalerna, samt möbler till trivselytor där teknologen i gemen har tillträde.
\subsubsection{Förvaltning}
Fonden förvaltas av sektionens styrelse.
\subsubsection{Avsättning}
\label{sec:lokalfond_avsattning}
\begin{itemize}
\item Minst Tio (10) \% av under verksamhetsåret influtna sektionsavgifter tillförs lokalfonden.
\item En av styrelsen budgeterad summa som godkänts av sektionsmötet
\item All avkastning ifrån lokalfonden tillförs lokalfonden.
\end{itemize}
\subsubsection{Uttag}
\begin{itemize}
\item Sektionsstyrelsen disponerar sjuttiofem (75) \% av årets tillförda kapital,
enligt \S\ref{sec:lokalfond_avsattning}, för basdrift av sektionslokalerna. 
\item Sektionsstyrelsen har rätt att besluta om ytterliggare uttag, dock skall detta redovisas inför nästkommande sektionsmöte. Vid detta sektionsmöte skall i sådana fall även genomförda eller planerade inköp, reparationer och underhåll redovisas.
\item Sektionsmötet äger rätt att vid välmotiverat behov göra uttag i strid med syftet i \S\ref{sec:lokalfond_syfte}.
\end{itemize}

\subsection{Bilfonden}
\subsubsection{Syfte}
\label{sec:bilfond_syfte}
Syftet med bilfonden är att bygga upp en buffert för bilinköp och minska belastningen av sektionens ekonomi vid omfattande skador.
\subsubsection{Förvaltning}
Fonden förvaltas av sektionens styrelse.
\subsubsection{Avsättning}
\begin{itemize}
\item En av styrelsens budgeterad summa som godkänts av sektionsmötet.
\item All avkastning ifrån bilfonden tillförs bilfonden.
\end{itemize}
\subsubsection{Uttag}
\begin{itemize}
\item Sektionsstyrelsen äger rätt att bevilja uttag för inköp av en ny bil.
\item Sektionsstyrelsen äger rätt att bevilja uttag för omfattande reparation av sektionensbil.
\item Sektionsmötet äger rätt att vid välmotiverat behov göra uttag i strid med syftet i \S\ref{sec:bilfond_syfte}.
\end{itemize}

\subsection{Idéfonden}
\subsubsection{Syfte}
\label{sec:idefond_syfte}
Syftet med idéfonden är att möjliggöra för datateknologer som önskar medel till arrangemang eller inventarier vars effekt är till gagn för hela teknologsektionens medlemmar. Kommittéer eller andra grupper under teknologsektionen som önskar medel för sådana ändamål är inte föremål för idéfonden.
\subsubsection{Förvaltning}
Idéfonden förvaltas av sektionsstyrelsen.
\subsubsection{Avsättning}
\begin{itemize}
\item En av styrelsen budgeterad summa som godkänts av sektionsmötet.
\item All avkastning ifrån idéfonden tillförs idéfonden.
\item Tio (10) \% av under verksamhetsåret influtna sektionsavgifter tillförs idéfonden.
\end{itemize}
\subsubsection{Uttag}
\begin{itemize}
  \item Sektionsstyrelsen har rätt att bevilja uttag ur fonden på belopp upp till totalt 0,25 prisbasbelopp per tillfälle. Uttag av belopp överstigande 0,25 prisbasbelopp skall godkännas av sektionsmöte innan medel utbetalas.
  \item Sektionsstyrelsen äger inte rätt att ta ut mer än 50\% av fondens totala värde per tillfälle.
  \item Sektionsmötet äger rätt att vid välmotiverat behov göra uttag över 0.25 prisbasbelopp eller i strid med syftet i \S\ref{sec:idefond_syfte}.
\subsubsection{Motstridig uttag}
Sektionsmötet kan besluta om att göra uttag i strid med idéfondens syfte.

\end{document}
